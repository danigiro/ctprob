\pdfminorversion=1
\documentclass[a4paper,12pt]{article}
\newcommand\hmmax{0}
\newcommand\bmmax{0}
\usepackage{amsmath}
\usepackage{amsthm}
\usepackage{amsfonts}
\usepackage{mathrsfs} 
\usepackage{amssymb}
\usepackage[dvipsnames]{xcolor}
%\usepackage{mathabx}
\usepackage{bm}
\usepackage{bbm}
\usepackage[bb=boondox]{mathalfa}
\usepackage{paralist}
\usepackage{natbib}
\usepackage{url}
\usepackage[textwidth=6.1in,textheight = 10in]{geometry}
\usepackage{placeins}
\usepackage[hidelinks]{hyperref}
\usepackage{multirow, float, textcmds, siunitx}
\usepackage{makecell}
\usepackage[textwidth=24mm, colorinlistoftodos, textsize=scriptsize, backgroundcolor=white, linecolor=black]{todonotes}
\usepackage{graphicx}
\usepackage{enumitem}
\usepackage{algorithm, algorithmicx, algpseudocode}
%\usepackage{newtxtext, newtxmath}
\usepackage{comment}
\usepackage{ulem}
\setlength{\parskip}{0cm}
\setlength{\parindent}{1em}
\usepackage[compact]{titlesec}
\titlespacing{\section}{0pt}{2ex}{1ex}
\titlespacing{\subsection}{0pt}{1ex}{0ex}
\titlespacing{\subsubsection}{0pt}{0.5ex}{0ex}

\usepackage[bottom,hang,flushmargin]{footmisc}

\makeatletter
\newcommand*\rel@kern[1]{\kern#1\dimexpr\macc@kerna}
\newcommand*\widebar[1]{%
  \begingroup
  \def\mathaccent##1##2{%
    \rel@kern{0.8}%
    \overline{\rel@kern{-0.8}\macc@nucleus\rel@kern{0.2}}%
    \rel@kern{-0.2}%
  }%
  \macc@depth\@ne
  \let\math@bgroup\@empty \let\math@egroup\macc@set@skewchar
  \mathsurround\z@ \frozen@everymath{\mathgroup\macc@group\relax}%
  \macc@set@skewchar\relax
  \let\mathaccentV\macc@nested@a
  \macc@nested@a\relax111{#1}%
  \endgroup
}
\makeatother

\usepackage{adjustbox}

%% LINE AND PAGE BREAKING
\sloppy
\clubpenalty = 10000
\widowpenalty = 10000
\brokenpenalty = 10000
\RequirePackage{microtype}

% NOTE: To produce blinded version, replace "0" with "1" below.
\newcommand{\blind}{0}

%\usepackage{setspace}
\usepackage{caption}
\captionsetup[figure]{labelfont={bf}, font = small, singlelinecheck=true}
\captionsetup[table]{labelfont={bf}, font = small, singlelinecheck=true}
\usepackage{subcaption}

\DeclareMathOperator*{\argmin}{arg\,min}
\usepackage{longtable}
\usepackage{booktabs}
\usepackage{array}
\newcolumntype{M}[1]{>{\centering\arraybackslash}m{#1}}
\newcolumntype{L}[1]{>{\raggedright\arraybackslash}m{#1}}
\newcolumntype{R}[1]{>{\raggedleft\arraybackslash}m{#1}}


\newcommand{\alphavet}{\mathbf{\alpha}}
\newcommand{\betavet}{\pmb{\beta}}
\newcommand{\epsvet}{\mathbf{\varepsilon}}
\newcommand{\etavet}{\pmb{\eta}}
\newcommand{\lambdavet}{\mathbf{\lambda}}
\newcommand{\Unovet}{\mathbf{1}}
\newcommand{\avet}{\textbf{a}}
\newcommand{\bvet}{\textbf{b}}
\newcommand{\cvet}{\textbf{c}}
\newcommand{\dvet}{\textbf{d}}
\newcommand{\evet}{\textbf{e}}
\newcommand{\pvet}{\textbf{p}}
\newcommand{\fvet}{\textbf{f}}
\newcommand{\tvet}{\textbf{t}}
\newcommand{\uvet}{\textbf{u}}
\newcommand{\vvet}{\textbf{v}}
\newcommand{\wvet}{\textbf{w}}
\newcommand{\xvet}{\textbf{x}}
\newcommand{\yvet}{\textbf{y}}
\newcommand{\zvet}{\textbf{z}}
\newcommand{\Avet}{\textbf{A}}
\newcommand{\Bvet}{\textbf{B}}
\newcommand{\Cvet}{\textbf{C}}
\newcommand{\Dvet}{\textbf{D}}
\newcommand{\Evet}{\textbf{E}}
\newcommand{\Fvet}{\textbf{F}}
\newcommand{\Gvet}{\textbf{G}}
\newcommand{\Hvet}{\textbf{H}}
\newcommand{\Ivet}{\textbf{I}}
\newcommand{\Jvet}{\textbf{J}}
\newcommand{\Kvet}{\textbf{K}}
\newcommand{\Lvet}{\textbf{L}}
\newcommand{\Mvet}{\textbf{M}}
\newcommand{\Nvet}{\textbf{N}}
\newcommand{\Pvet}{\textbf{P}}
\newcommand{\Qvet}{\textbf{Q}}
\newcommand{\Rvet}{\textbf{R}}
\newcommand{\Svet}{\textbf{S}}
\newcommand{\Tvet}{\textbf{T}}
\newcommand{\Uvet}{\textbf{U}}
\newcommand{\Wvet}{\textbf{W}}
\newcommand{\Xvet}{\textbf{X}}
\newcommand{\Yvet}{\textbf{Y}}
\newcommand{\Zvet}{\textbf{Z}}
\newcommand{\Zerovet}{\textbf{0}}
\bmdefine{\xhvet}{\mathsf{x}}
\bmdefine{\yhvet}{\mathsf{y}}
\bmdefine{\Chvet}{\mathsf{C}}
\bmdefine{\Ghvet}{\mathsf{G}}
\bmdefine{\Jhvet}{\mathsf{J}}
\bmdefine{\Shvet}{\mathsf{S}}
\bmdefine{\Uhvet}{\mathsf{U}}
\bmdefine{\Gammavet}{\mathsf{\Gamma}}
%\bmdefine{\Omegavet}{\mathsf{\Omega}}
\newcommand{\Omegavet}{\mathbf{\Omega}}
\newcommand{\Sigmavet}{\mathbf{\Sigma}}
\bmdefine{\Thetavet}{\mathsf{\Theta}}
\bmdefine{\ahvet}{\mathfrak{a}}
\bmdefine{\bhvet}{\mathfrak{b}}

\definecolor{mybluehl}{HTML}{cbd3ff}

% theorem
\makeatletter
\def\@endtheorem{\endtrivlist}
\makeatother

%% tikz
%% Packages to draw hierarchies
\usepackage{tikz}
\usepackage{forest}

\usetikzlibrary{arrows,shapes,positioning,shadows,trees}
\usetikzlibrary{matrix, decorations.pathreplacing, arrows, calc, fit, arrows.meta, decorations.pathmorphing, decorations.markings}


\tikzset{
  basic/.style  = {draw, text width=2cm, drop shadow, font=\sffamily, rectangle},
  root/.style   = {basic, rounded corners=2pt, thin, align=center,
                   fill=green!30},
  level 2/.style = {basic, rounded corners=6pt, thin,align=center, fill=green!60,
                   text width=4em},
  level 3/.style = {basic, thin, align=left, fill=pink!60, text width=1.5em}
}
\newcommand{\relation}[3]
{
	\draw (#3.south) -- +(0,-#1) -| ($ (#2.north) $)
}
\newcommand{\relationW}[2]
{
	\draw (#2.west) -| ($ (#1.north) $)
}
\newcommand{\relationE}[2]
{
	\draw (#2.east) -| ($ (#1.north) $)
}

\newcommand{\relationD}[3]
{
	\draw (#3.east) -- +(#1,0) |- (#2.west)
}

\pgfdeclareimage[height=0.85cm]{ngreen}{fig/boot/ngreen.pdf}
\pgfdeclareimage[height=0.85cm]{nblue}{fig/boot/nblue.pdf}
\pgfdeclareimage[height=0.85cm]{nred}{fig/boot/nred.pdf}
\pgfdeclareimage[height=0.85cm]{nblack}{fig/boot/nblack.pdf}

\theoremstyle{definition}
\newtheorem{definition}{Definition}[section]
\newtheorem{theorem}{Theorem}[section]

% Title page
\makeatletter  
\newcommand{\maketitleblind}{\begingroup%
\if1\blind
{
\clearpage\maketitle
\thispagestyle{empty}
\vfill
%\vskip2cm
%\noindent\textit{\large\textbf{Preliminary Working Draft}}\\
%\noindent\textbf{Please do not quote or cite without authors' permission}
\vfill
\newpage
\setcounter{page}{1}
}\fi

\if0\blind
{
\begin{center}%
  \let \footnote \thanks
    {\LARGE \@title \par}%
    \vskip 1.5em%
    {\large \@date}%
  \end{center}
  \bigskip
} \fi
\endgroup}
\makeatother

% Authors code
\usepackage[affil-it]{authblk}
\setlength{\affilsep}{0em}
\newcommand{\email}[1]{\affil{Email: {\upshape\href{mailto:#1}{#1}}}}
\renewcommand\Affilfont{\itshape\normalsize}
  
% Abstract code
\makeatletter
\renewenvironment{abstract}{%
    \if@twocolumn
      \section*{\abstractname}%
    \else %
      \begin{center}%
        {\bfseries \large\abstractname\vspace{\z@}}%
      \end{center}%
      \quotation
    \fi}
    {\if@twocolumn\else\endquotation\fi}
\makeatother

%% Settings
\title{\bf Cross-temporal Probabilistic Forecast Reconciliation: online appendix}
\author{Tommaso Di Fonzo}
\affil{Department of Statistical Sciences, University of Padova}
\email{tommaso.difonzo@unipd.it}
\author{Daniele Girolimetto}
\affil{Department of Statistical Sciences, University of Padova}
\email{daniele.girolimetto@phd.unipd.it}



\begin{document}

\def\spacingset#1{\renewcommand{\baselinestretch}{#1}\small\normalsize}
\spacingset{1.1}
  
\thispagestyle{empty} \clearpage\maketitleblind

\spacingset{1.5}
\tableofcontents
%\listoftables
\appendix

\section{AR2}

\begin{table}[H]
\centering
\begingroup
\spacingset{1}
\fontsize{9}{11}\selectfont

\begin{tabular}[t]{c|>{}cccc>{}c|ccccc}
\toprule
\multicolumn{1}{c}{\textbf{}} & \multicolumn{10}{c}{\textbf{Base forecasts' sample approach}} \\
\cmidrule(l{0pt}r{0pt}){2-11}
\multicolumn{1}{c}{\makecell[c]{\bfseries Reconciliation\\\bfseries approach}} & \multicolumn{1}{c}{ctjb} & \multicolumn{4}{c}{\makecell[c]{Gaussian approach\textsuperscript{*}}} & \multicolumn{1}{c}{ctjb} & \multicolumn{4}{c}{\makecell[c]{Gaussian approach\textsuperscript{*}}} \\
\multicolumn{1}{c}{} &  & G & B & H & \multicolumn{1}{c}{HB} &  & G & B & H & HB\\
\midrule
\addlinespace[0.3em]
\multicolumn{1}{c}{} & \multicolumn{5}{c}{\textbf{$\forall k \in \{12,6,4,3,2,1\}$}} & \multicolumn{5}{c}{\textbf{$k = 1$}}\\
base & \textcolor{black}{1.000} & \textcolor{black}{\textbf{0.971}} & \textcolor{black}{0.972} & \textcolor{black}{\textbf{0.971}} & \textcolor{black}{0.972} & \textcolor{black}{1.000} & \textcolor{black}{\textbf{0.972}} & \textcolor{black}{0.971} & \textcolor{black}{\textbf{0.972}} & \textcolor{black}{0.971}\\
ct$(bu)$ & \textcolor{red}{1.321} & \textcolor{red}{1.017} & \textcolor{red}{1.018} & \textcolor{red}{1.017} & \textcolor{red}{1.017} & \textcolor{red}{1.077} & \textcolor{black}{0.983} & \textcolor{black}{0.983} & \textcolor{black}{0.983} & \textcolor{black}{0.983}\\
ct$(shr_{cs}, bu_{te})$ & \textcolor{red}{1.057} & \textcolor{red}{1.013} & \textcolor{black}{\textbf{0.971}} & \textcolor{red}{1.013} & \textcolor{black}{0.971} & \textcolor{black}{0.976} & \textcolor{black}{0.987} & \textcolor{black}{\textbf{0.961}} & \textcolor{black}{0.988} & \textcolor{black}{0.961}\\
ct$(wls_{cs}, bu_{te})$ & \textcolor{red}{1.082} & \textcolor{red}{1.020} & \textcolor{black}{0.978} & \textcolor{red}{1.020} & \textcolor{black}{0.978} & \textcolor{black}{0.986} & \textcolor{black}{0.991} & \textcolor{black}{0.964} & \textcolor{black}{0.991} & \textcolor{black}{0.964}\\
oct$(wlsv)$ & \textcolor{black}{0.987} & \textcolor{red}{1.080} & \textcolor{red}{1.041} & \textcolor{black}{0.992} & \textcolor{black}{0.958} & \textcolor{black}{0.952} & \textcolor{red}{1.004} & \textcolor{black}{0.969} & \textcolor{black}{0.978} & \textcolor{black}{0.956}\\
oct$(bdshr)$ & \textcolor{black}{\textbf{0.975}} & \textcolor{red}{1.072} & \textcolor{red}{1.032} & \textcolor{black}{0.985} & \textcolor{blue}{\textbf{0.950}} & \textcolor{blue}{\textbf{0.949}} & \textcolor{black}{0.999} & \textcolor{black}{0.965} & \textcolor{black}{0.975} & \textcolor{black}{\textbf{0.952}}\\
\addlinespace[0.3em]
\multicolumn{1}{c}{} & \multicolumn{5}{c}{\textbf{$k = 2$}} & \multicolumn{5}{c}{\textbf{$k = 3$}}\\
base & \textcolor{black}{1.000} & \textcolor{black}{\textbf{0.969}} & \textcolor{black}{0.969} & \textcolor{black}{\textbf{0.968}} & \textcolor{black}{0.968} & \textcolor{black}{1.000} & \textcolor{black}{\textbf{0.971}} & \textcolor{black}{\textbf{0.970}} & \textcolor{black}{\textbf{0.969}} & \textcolor{black}{0.970}\\
ct$(bu)$ & \textcolor{red}{1.189} & \textcolor{black}{1.000} & \textcolor{red}{1.000} & \textcolor{red}{1.000} & \textcolor{red}{1.000} & \textcolor{red}{1.273} & \textcolor{red}{1.013} & \textcolor{red}{1.013} & \textcolor{red}{1.013} & \textcolor{red}{1.013}\\
ct$(shr_{cs}, bu_{te})$ & \textcolor{red}{1.015} & \textcolor{red}{1.004} & \textcolor{black}{\textbf{0.968}} & \textcolor{red}{1.004} & \textcolor{black}{0.968} & \textcolor{red}{1.041} & \textcolor{red}{1.013} & \textcolor{black}{0.973} & \textcolor{red}{1.014} & \textcolor{black}{0.973}\\
ct$(wls_{cs}, bu_{te})$ & \textcolor{red}{1.031} & \textcolor{red}{1.009} & \textcolor{black}{0.973} & \textcolor{red}{1.009} & \textcolor{black}{0.973} & \textcolor{red}{1.062} & \textcolor{red}{1.020} & \textcolor{black}{0.979} & \textcolor{red}{1.020} & \textcolor{black}{0.979}\\
oct$(wlsv)$ & \textcolor{black}{0.972} & \textcolor{red}{1.064} & \textcolor{red}{1.021} & \textcolor{black}{0.987} & \textcolor{black}{0.958} & \textcolor{black}{0.983} & \textcolor{red}{1.083} & \textcolor{red}{1.041} & \textcolor{black}{0.993} & \textcolor{black}{0.960}\\
oct$(bdshr)$ & \textcolor{black}{\textbf{0.964}} & \textcolor{red}{1.057} & \textcolor{red}{1.015} & \textcolor{black}{0.983} & \textcolor{blue}{\textbf{0.953}} & \textcolor{black}{\textbf{0.972}} & \textcolor{red}{1.075} & \textcolor{red}{1.033} & \textcolor{black}{0.988} & \textcolor{blue}{\textbf{0.955}}\\
\addlinespace[0.3em]
\multicolumn{1}{c}{} & \multicolumn{5}{c}{\textbf{$k = 4$}} & \multicolumn{5}{c}{\textbf{$k = 6$}}\\
base & \textcolor{black}{1.000} & \textcolor{black}{\textbf{0.973}} & \textcolor{black}{\textbf{0.973}} & \textcolor{black}{\textbf{0.971}} & \textcolor{black}{0.973} & \textcolor{black}{1.000} & \textcolor{black}{\textbf{0.976}} & \textcolor{black}{0.977} & \textcolor{black}{\textbf{0.975}} & \textcolor{black}{0.977}\\
ct$(bu)$ & \textcolor{red}{1.340} & \textcolor{red}{1.021} & \textcolor{red}{1.021} & \textcolor{red}{1.021} & \textcolor{red}{1.021} & \textcolor{red}{1.450} & \textcolor{red}{1.032} & \textcolor{red}{1.033} & \textcolor{red}{1.032} & \textcolor{red}{1.033}\\
ct$(shr_{cs}, bu_{te})$ & \textcolor{red}{1.061} & \textcolor{red}{1.018} & \textcolor{black}{0.974} & \textcolor{red}{1.018} & \textcolor{black}{0.974} & \textcolor{red}{1.094} & \textcolor{red}{1.023} & \textcolor{black}{\textbf{0.974}} & \textcolor{red}{1.024} & \textcolor{black}{0.974}\\
ct$(wls_{cs}, bu_{te})$ & \textcolor{red}{1.087} & \textcolor{red}{1.025} & \textcolor{black}{0.981} & \textcolor{red}{1.026} & \textcolor{black}{0.981} & \textcolor{red}{1.127} & \textcolor{red}{1.033} & \textcolor{black}{0.984} & \textcolor{red}{1.033} & \textcolor{black}{0.984}\\
oct$(wlsv)$ & \textcolor{black}{0.990} & \textcolor{red}{1.100} & \textcolor{red}{1.059} & \textcolor{black}{0.996} & \textcolor{black}{0.959} & \textcolor{red}{1.001} & \textcolor{red}{1.115} & \textcolor{red}{1.076} & \textcolor{black}{0.998} & \textcolor{black}{0.958}\\
oct$(bdshr)$ & \textcolor{black}{\textbf{0.977}} & \textcolor{red}{1.091} & \textcolor{red}{1.049} & \textcolor{black}{0.989} & \textcolor{blue}{\textbf{0.952}} & \textcolor{black}{\textbf{0.985}} & \textcolor{red}{1.103} & \textcolor{red}{1.064} & \textcolor{black}{0.989} & \textcolor{blue}{\textbf{0.949}}\\
\addlinespace[0.3em]
\multicolumn{1}{c}{} & \multicolumn{5}{c}{\textbf{$k = 12$}} & \multicolumn{5}{c}{}\\
base & \textcolor{black}{\textbf{1.000}} & \textcolor{black}{\textbf{0.968}} & \textcolor{black}{\textbf{0.969}} & \textcolor{black}{\textbf{0.969}} & \textcolor{black}{0.971} &  &  &  &  & \\
ct$(bu)$ & \textcolor{red}{1.675} & \textcolor{red}{1.056} & \textcolor{red}{1.057} & \textcolor{red}{1.057} & \textcolor{red}{1.057} &  &  &  &  & \\
ct$(shr_{cs}, bu_{te})$ & \textcolor{red}{1.163} & \textcolor{red}{1.032} & \textcolor{black}{0.974} & \textcolor{red}{1.033} & \textcolor{black}{0.974} &  &  &  &  & \\
ct$(wls_{cs}, bu_{te})$ & \textcolor{red}{1.212} & \textcolor{red}{1.043} & \textcolor{black}{0.987} & \textcolor{red}{1.044} & \textcolor{black}{0.987} &  &  &  &  & \\
oct$(wlsv)$ & \textcolor{red}{1.025} & \textcolor{red}{1.122} & \textcolor{red}{1.085} & \textcolor{red}{1.001} & \textcolor{black}{0.954} &  &  &  &  & \\
oct$(bdshr)$ & \textcolor{red}{1.002} & \textcolor{red}{1.110} & \textcolor{red}{1.071} & \textcolor{black}{0.989} & \textcolor{blue}{\textbf{0.941}} &  &  &  &  & \\
\bottomrule
\multicolumn{11}{l}{\rule{0pt}{1em}\rule{0pt}{2em}\textsuperscript{*}\makecell[l]{The Gaussian method employs a shrinkage covariance matrix and includes four techniques (G, B, H, HB)\\ with multi-step residuals.}}\\
\end{tabular}

\endgroup
\caption{CRPS skill score presented in equation (18) and (19) of the paper. The smaller this value, the more accurate the forecast. Approaches that performed worse than the benchmark model (base, $G$) are highlighted in red, the best for each column is marked in bold and in blue the lowest value. The notation used to refer to the reconciliation and base forecast samples is explained in more details in Section 7.1 of the paper.}
\end{table}

\begin{table}[H]
\centering
\begingroup
\spacingset{1}
\fontsize{9}{11}\selectfont

\begin{tabular}[t]{c|>{}cccc>{}c|ccccc}
\toprule
\multicolumn{1}{c}{\textbf{}} & \multicolumn{10}{c}{\textbf{Base forecasts' sample approach}} \\
\cmidrule(l{0pt}r{0pt}){2-11}
\multicolumn{1}{c}{\makecell[c]{\bfseries Reconciliation\\\bfseries approach}} & \multicolumn{1}{c}{ctjb} & \multicolumn{4}{c}{\makecell[c]{Gaussian approach\textsuperscript{*}}} & \multicolumn{1}{c}{ctjb} & \multicolumn{4}{c}{\makecell[c]{Gaussian approach\textsuperscript{*}}} \\
\multicolumn{1}{c}{} &  & G & B & H & \multicolumn{1}{c}{HB} &  & G & B & H & HB\\
\midrule
\addlinespace[0.3em]
\multicolumn{1}{c}{} & \multicolumn{5}{c}{\textbf{$\forall k \in \{12,6,4,3,2,1\}$}} & \multicolumn{5}{c}{\textbf{$k = 1$}}\\
base & \textcolor{black}{\textbf{1.000}} & \textcolor{black}{\textbf{0.958}} & \textcolor{black}{0.984} & \textcolor{black}{\textbf{0.972}} & \textcolor{black}{0.992} & \textcolor{black}{\textbf{1.000}} & \textcolor{black}{\textbf{0.954}} & \textcolor{black}{0.958} & \textcolor{black}{\textbf{0.954}} & \textcolor{black}{0.958}\\
ct$(bu)$ & \textcolor{red}{2.427} & \textcolor{red}{1.040} & \textcolor{red}{1.042} & \textcolor{red}{1.040} & \textcolor{red}{1.041} & \textcolor{red}{1.759} & \textcolor{red}{1.001} & \textcolor{red}{1.002} & \textcolor{red}{1.002} & \textcolor{red}{1.002}\\
ct$(shr_{cs}, bu_{te})$ & \textcolor{red}{1.243} & \textcolor{black}{0.988} & \textcolor{black}{\textbf{0.913}} & \textcolor{black}{0.990} & \textcolor{blue}{\textbf{0.913}} & \textcolor{red}{1.098} & \textcolor{red}{1.011} & \textcolor{black}{\textbf{0.938}} & \textcolor{red}{1.013} & \textcolor{blue}{\textbf{0.938}}\\
ct$(wls_{cs}, bu_{te})$ & \textcolor{red}{1.348} & \textcolor{red}{1.020} & \textcolor{black}{0.940} & \textcolor{red}{1.022} & \textcolor{black}{0.939} & \textcolor{red}{1.151} & \textcolor{red}{1.028} & \textcolor{black}{0.950} & \textcolor{red}{1.030} & \textcolor{black}{0.950}\\
oct$(wlsv)$ & \textcolor{red}{1.132} & \textcolor{red}{1.137} & \textcolor{red}{1.065} & \textcolor{red}{1.059} & \textcolor{black}{0.969} & \textcolor{red}{1.050} & \textcolor{red}{1.078} & \textcolor{black}{0.989} & \textcolor{red}{1.043} & \textcolor{black}{0.960}\\
oct$(bdshr)$ & \textcolor{red}{1.047} & \textcolor{red}{1.085} & \textcolor{red}{1.013} & \textcolor{red}{1.011} & \textcolor{black}{0.927} & \textcolor{red}{1.009} & \textcolor{red}{1.050} & \textcolor{black}{0.966} & \textcolor{red}{1.019} & \textcolor{black}{0.942}\\
\addlinespace[0.3em]
\multicolumn{1}{c}{} & \multicolumn{5}{c}{\textbf{$k = 2$}} & \multicolumn{5}{c}{\textbf{$k = 3$}}\\
base & \textcolor{black}{\textbf{1.000}} & \textcolor{black}{\textbf{0.960}} & \textcolor{black}{0.971} & \textcolor{black}{\textbf{0.958}} & \textcolor{black}{0.972} & \textcolor{black}{\textbf{1.000}} & \textcolor{black}{\textbf{0.963}} & \textcolor{black}{0.981} & \textcolor{black}{\textbf{0.966}} & \textcolor{black}{0.986}\\
ct$(bu)$ & \textcolor{red}{2.176} & \textcolor{red}{1.035} & \textcolor{red}{1.036} & \textcolor{red}{1.035} & \textcolor{red}{1.035} & \textcolor{red}{2.428} & \textcolor{red}{1.042} & \textcolor{red}{1.044} & \textcolor{red}{1.042} & \textcolor{red}{1.043}\\
ct$(shr_{cs}, bu_{te})$ & \textcolor{red}{1.192} & \textcolor{red}{1.020} & \textcolor{black}{\textbf{0.942}} & \textcolor{red}{1.021} & \textcolor{blue}{\textbf{0.942}} & \textcolor{red}{1.245} & \textcolor{red}{1.009} & \textcolor{black}{\textbf{0.931}} & \textcolor{red}{1.011} & \textcolor{blue}{\textbf{0.931}}\\
ct$(wls_{cs}, bu_{te})$ & \textcolor{red}{1.275} & \textcolor{red}{1.045} & \textcolor{black}{0.962} & \textcolor{red}{1.047} & \textcolor{black}{0.962} & \textcolor{red}{1.348} & \textcolor{red}{1.038} & \textcolor{black}{0.955} & \textcolor{red}{1.040} & \textcolor{black}{0.954}\\
oct$(wlsv)$ & \textcolor{red}{1.110} & \textcolor{red}{1.149} & \textcolor{red}{1.065} & \textcolor{red}{1.070} & \textcolor{black}{0.979} & \textcolor{red}{1.142} & \textcolor{red}{1.160} & \textcolor{red}{1.082} & \textcolor{red}{1.073} & \textcolor{black}{0.981}\\
oct$(bdshr)$ & \textcolor{red}{1.045} & \textcolor{red}{1.105} & \textcolor{red}{1.024} & \textcolor{red}{1.033} & \textcolor{black}{0.949} & \textcolor{red}{1.060} & \textcolor{red}{1.109} & \textcolor{red}{1.032} & \textcolor{red}{1.029} & \textcolor{black}{0.943}\\
\addlinespace[0.3em]
\multicolumn{1}{c}{} & \multicolumn{5}{c}{\textbf{$k = 4$}} & \multicolumn{5}{c}{\textbf{$k = 6$}}\\
base & \textcolor{black}{\textbf{1.000}} & \textcolor{black}{\textbf{0.962}} & \textcolor{black}{0.987} & \textcolor{black}{\textbf{0.973}} & \textcolor{black}{0.996} & \textcolor{black}{\textbf{1.000}} & \textcolor{black}{\textbf{0.963}} & \textcolor{black}{0.998} & \textcolor{black}{\textbf{0.984}} & \textcolor{red}{1.011}\\
ct$(bu)$ & \textcolor{red}{2.585} & \textcolor{red}{1.052} & \textcolor{red}{1.054} & \textcolor{red}{1.053} & \textcolor{red}{1.053} & \textcolor{red}{2.849} & \textcolor{red}{1.083} & \textcolor{red}{1.085} & \textcolor{red}{1.083} & \textcolor{red}{1.084}\\
ct$(shr_{cs}, bu_{te})$ & \textcolor{red}{1.277} & \textcolor{red}{1.000} & \textcolor{blue}{\textbf{0.923}} & \textcolor{red}{1.002} & \textcolor{black}{\textbf{0.923}} & \textcolor{red}{1.339} & \textcolor{black}{0.999} & \textcolor{black}{\textbf{0.921}} & \textcolor{red}{1.000} & \textcolor{blue}{\textbf{0.920}}\\
ct$(wls_{cs}, bu_{te})$ & \textcolor{red}{1.392} & \textcolor{red}{1.032} & \textcolor{black}{0.950} & \textcolor{red}{1.035} & \textcolor{black}{0.950} & \textcolor{red}{1.473} & \textcolor{red}{1.038} & \textcolor{black}{0.954} & \textcolor{red}{1.040} & \textcolor{black}{0.954}\\
oct$(wlsv)$ & \textcolor{red}{1.157} & \textcolor{red}{1.167} & \textcolor{red}{1.097} & \textcolor{red}{1.075} & \textcolor{black}{0.982} & \textcolor{red}{1.192} & \textcolor{red}{1.187} & \textcolor{red}{1.124} & \textcolor{red}{1.090} & \textcolor{black}{0.995}\\
oct$(bdshr)$ & \textcolor{red}{1.065} & \textcolor{red}{1.112} & \textcolor{red}{1.041} & \textcolor{red}{1.025} & \textcolor{black}{0.939} & \textcolor{red}{1.084} & \textcolor{red}{1.121} & \textcolor{red}{1.058} & \textcolor{red}{1.029} & \textcolor{black}{0.940}\\
\addlinespace[0.3em]
\multicolumn{1}{c}{} & \multicolumn{5}{c}{\textbf{$k = 12$}} & \multicolumn{5}{c}{}\\
base & \textcolor{black}{\textbf{1.000}} & \textcolor{black}{0.948} & \textcolor{red}{1.010} & \textcolor{red}{1.002} & \textcolor{red}{1.033} &  &  &  &  & \\
ct$(bu)$ & \textcolor{red}{2.990} & \textcolor{red}{1.028} & \textcolor{red}{1.031} & \textcolor{red}{1.029} & \textcolor{red}{1.029} &  &  &  &  & \\
ct$(shr_{cs}, bu_{te})$ & \textcolor{red}{1.326} & \textcolor{black}{\textbf{0.897}} & \textcolor{black}{\textbf{0.830}} & \textcolor{black}{\textbf{0.899}} & \textcolor{blue}{\textbf{0.830}} &  &  &  &  & \\
ct$(wls_{cs}, bu_{te})$ & \textcolor{red}{1.477} & \textcolor{black}{0.942} & \textcolor{black}{0.872} & \textcolor{black}{0.945} & \textcolor{black}{0.870} &  &  &  &  & \\
oct$(wlsv)$ & \textcolor{red}{1.149} & \textcolor{red}{1.089} & \textcolor{red}{1.041} & \textcolor{red}{1.006} & \textcolor{black}{0.922} &  &  &  &  & \\
oct$(bdshr)$ & \textcolor{red}{1.021} & \textcolor{red}{1.015} & \textcolor{black}{0.964} & \textcolor{black}{0.935} & \textcolor{black}{0.855} &  &  &  &  & \\
\bottomrule
\multicolumn{11}{l}{\rule{0pt}{1em}\rule{0pt}{2em}\textsuperscript{*}\makecell[l]{The Gaussian method employs a shrinkage covariance matrix and includes four techniques (G, B, H, HB)\\ with multi-step residuals.}}\\
\end{tabular}

\endgroup
\caption{ES skill score presented in equation (18) and (19) of the paper. The smaller this value, the more accurate the forecast. Approaches that performed worse than the benchmark model (base, $G$) are highlighted in red, the best for each column is marked in bold and in blue the lowest value. The notation used to refer to the reconciliation and base forecast samples is explained in more details in Section 7.1 of the paper.}
\end{table}

\section{AusGDP}

\begin{table}[H]
\centering
\begingroup
\spacingset{1}
\fontsize{9}{11}\selectfont

\begin{tabular}[t]{c|>{}cccc>{}c|ccccc}
\toprule
\multicolumn{1}{c}{\textbf{}} & \multicolumn{10}{c}{\textbf{Base forecasts' sample approach}} \\
\cmidrule(l{0pt}r{0pt}){2-11}
\multicolumn{1}{c}{\makecell[c]{\bfseries Reconciliation\\\bfseries approach}} & \multicolumn{1}{c}{ctjb} & \multicolumn{4}{c}{\makecell[c]{Gaussian approach\textsuperscript{*}}} & \multicolumn{1}{c}{ctjb} & \multicolumn{4}{c}{\makecell[c]{Gaussian approach\textsuperscript{*}}} \\
\multicolumn{1}{c}{} &  & G & B & H & \multicolumn{1}{c}{HB} &  & G & B & H & HB\\
\midrule
\addlinespace[0.3em]
\multicolumn{1}{c}{} & \multicolumn{5}{c}{\textbf{$\forall k \in \{12,6,4,3,2,1\}$}} & \multicolumn{5}{c}{\textbf{$k = 1$}}\\
base & \textcolor{black}{1.000} & \textcolor{black}{\textbf{0.971}} & \textcolor{black}{0.972} & \textcolor{black}{\textbf{0.971}} & \textcolor{black}{0.972} & \textcolor{black}{1.000} & \textcolor{black}{\textbf{0.972}} & \textcolor{black}{0.971} & \textcolor{black}{\textbf{0.972}} & \textcolor{black}{0.971}\\
ct$(bu)$ & \textcolor{red}{1.321} & \textcolor{red}{1.017} & \textcolor{red}{1.018} & \textcolor{red}{1.017} & \textcolor{red}{1.017} & \textcolor{red}{1.077} & \textcolor{black}{0.983} & \textcolor{black}{0.983} & \textcolor{black}{0.983} & \textcolor{black}{0.983}\\
ct$(shr_{cs}, bu_{te})$ & \textcolor{red}{1.057} & \textcolor{red}{1.013} & \textcolor{black}{\textbf{0.971}} & \textcolor{red}{1.013} & \textcolor{black}{0.971} & \textcolor{black}{0.976} & \textcolor{black}{0.987} & \textcolor{black}{\textbf{0.961}} & \textcolor{black}{0.988} & \textcolor{black}{0.961}\\
ct$(wls_{cs}, bu_{te})$ & \textcolor{red}{1.082} & \textcolor{red}{1.020} & \textcolor{black}{0.978} & \textcolor{red}{1.020} & \textcolor{black}{0.978} & \textcolor{black}{0.986} & \textcolor{black}{0.991} & \textcolor{black}{0.964} & \textcolor{black}{0.991} & \textcolor{black}{0.964}\\
oct$(wlsv)$ & \textcolor{black}{0.987} & \textcolor{red}{1.080} & \textcolor{red}{1.041} & \textcolor{black}{0.992} & \textcolor{black}{0.958} & \textcolor{black}{0.952} & \textcolor{red}{1.004} & \textcolor{black}{0.969} & \textcolor{black}{0.978} & \textcolor{black}{0.956}\\
oct$(bdshr)$ & \textcolor{black}{\textbf{0.975}} & \textcolor{red}{1.072} & \textcolor{red}{1.032} & \textcolor{black}{0.985} & \textcolor{blue}{\textbf{0.950}} & \textcolor{blue}{\textbf{0.949}} & \textcolor{black}{0.999} & \textcolor{black}{0.965} & \textcolor{black}{0.975} & \textcolor{black}{\textbf{0.952}}\\
\addlinespace[0.3em]
\multicolumn{1}{c}{} & \multicolumn{5}{c}{\textbf{$k = 2$}} & \multicolumn{5}{c}{\textbf{$k = 3$}}\\
base & \textcolor{black}{1.000} & \textcolor{black}{\textbf{0.969}} & \textcolor{black}{0.969} & \textcolor{black}{\textbf{0.968}} & \textcolor{black}{0.968} & \textcolor{black}{1.000} & \textcolor{black}{\textbf{0.971}} & \textcolor{black}{\textbf{0.970}} & \textcolor{black}{\textbf{0.969}} & \textcolor{black}{0.970}\\
ct$(bu)$ & \textcolor{red}{1.189} & \textcolor{black}{1.000} & \textcolor{red}{1.000} & \textcolor{red}{1.000} & \textcolor{red}{1.000} & \textcolor{red}{1.273} & \textcolor{red}{1.013} & \textcolor{red}{1.013} & \textcolor{red}{1.013} & \textcolor{red}{1.013}\\
ct$(shr_{cs}, bu_{te})$ & \textcolor{red}{1.015} & \textcolor{red}{1.004} & \textcolor{black}{\textbf{0.968}} & \textcolor{red}{1.004} & \textcolor{black}{0.968} & \textcolor{red}{1.041} & \textcolor{red}{1.013} & \textcolor{black}{0.973} & \textcolor{red}{1.014} & \textcolor{black}{0.973}\\
ct$(wls_{cs}, bu_{te})$ & \textcolor{red}{1.031} & \textcolor{red}{1.009} & \textcolor{black}{0.973} & \textcolor{red}{1.009} & \textcolor{black}{0.973} & \textcolor{red}{1.062} & \textcolor{red}{1.020} & \textcolor{black}{0.979} & \textcolor{red}{1.020} & \textcolor{black}{0.979}\\
oct$(wlsv)$ & \textcolor{black}{0.972} & \textcolor{red}{1.064} & \textcolor{red}{1.021} & \textcolor{black}{0.987} & \textcolor{black}{0.958} & \textcolor{black}{0.983} & \textcolor{red}{1.083} & \textcolor{red}{1.041} & \textcolor{black}{0.993} & \textcolor{black}{0.960}\\
oct$(bdshr)$ & \textcolor{black}{\textbf{0.964}} & \textcolor{red}{1.057} & \textcolor{red}{1.015} & \textcolor{black}{0.983} & \textcolor{blue}{\textbf{0.953}} & \textcolor{black}{\textbf{0.972}} & \textcolor{red}{1.075} & \textcolor{red}{1.033} & \textcolor{black}{0.988} & \textcolor{blue}{\textbf{0.955}}\\
\addlinespace[0.3em]
\multicolumn{1}{c}{} & \multicolumn{5}{c}{\textbf{$k = 4$}} & \multicolumn{5}{c}{\textbf{$k = 6$}}\\
base & \textcolor{black}{1.000} & \textcolor{black}{\textbf{0.973}} & \textcolor{black}{\textbf{0.973}} & \textcolor{black}{\textbf{0.971}} & \textcolor{black}{0.973} & \textcolor{black}{1.000} & \textcolor{black}{\textbf{0.976}} & \textcolor{black}{0.977} & \textcolor{black}{\textbf{0.975}} & \textcolor{black}{0.977}\\
ct$(bu)$ & \textcolor{red}{1.340} & \textcolor{red}{1.021} & \textcolor{red}{1.021} & \textcolor{red}{1.021} & \textcolor{red}{1.021} & \textcolor{red}{1.450} & \textcolor{red}{1.032} & \textcolor{red}{1.033} & \textcolor{red}{1.032} & \textcolor{red}{1.033}\\
ct$(shr_{cs}, bu_{te})$ & \textcolor{red}{1.061} & \textcolor{red}{1.018} & \textcolor{black}{0.974} & \textcolor{red}{1.018} & \textcolor{black}{0.974} & \textcolor{red}{1.094} & \textcolor{red}{1.023} & \textcolor{black}{\textbf{0.974}} & \textcolor{red}{1.024} & \textcolor{black}{0.974}\\
ct$(wls_{cs}, bu_{te})$ & \textcolor{red}{1.087} & \textcolor{red}{1.025} & \textcolor{black}{0.981} & \textcolor{red}{1.026} & \textcolor{black}{0.981} & \textcolor{red}{1.127} & \textcolor{red}{1.033} & \textcolor{black}{0.984} & \textcolor{red}{1.033} & \textcolor{black}{0.984}\\
oct$(wlsv)$ & \textcolor{black}{0.990} & \textcolor{red}{1.100} & \textcolor{red}{1.059} & \textcolor{black}{0.996} & \textcolor{black}{0.959} & \textcolor{red}{1.001} & \textcolor{red}{1.115} & \textcolor{red}{1.076} & \textcolor{black}{0.998} & \textcolor{black}{0.958}\\
oct$(bdshr)$ & \textcolor{black}{\textbf{0.977}} & \textcolor{red}{1.091} & \textcolor{red}{1.049} & \textcolor{black}{0.989} & \textcolor{blue}{\textbf{0.952}} & \textcolor{black}{\textbf{0.985}} & \textcolor{red}{1.103} & \textcolor{red}{1.064} & \textcolor{black}{0.989} & \textcolor{blue}{\textbf{0.949}}\\
\addlinespace[0.3em]
\multicolumn{1}{c}{} & \multicolumn{5}{c}{\textbf{$k = 12$}} & \multicolumn{5}{c}{}\\
base & \textcolor{black}{\textbf{1.000}} & \textcolor{black}{\textbf{0.968}} & \textcolor{black}{\textbf{0.969}} & \textcolor{black}{\textbf{0.969}} & \textcolor{black}{0.971} &  &  &  &  & \\
ct$(bu)$ & \textcolor{red}{1.675} & \textcolor{red}{1.056} & \textcolor{red}{1.057} & \textcolor{red}{1.057} & \textcolor{red}{1.057} &  &  &  &  & \\
ct$(shr_{cs}, bu_{te})$ & \textcolor{red}{1.163} & \textcolor{red}{1.032} & \textcolor{black}{0.974} & \textcolor{red}{1.033} & \textcolor{black}{0.974} &  &  &  &  & \\
ct$(wls_{cs}, bu_{te})$ & \textcolor{red}{1.212} & \textcolor{red}{1.043} & \textcolor{black}{0.987} & \textcolor{red}{1.044} & \textcolor{black}{0.987} &  &  &  &  & \\
oct$(wlsv)$ & \textcolor{red}{1.025} & \textcolor{red}{1.122} & \textcolor{red}{1.085} & \textcolor{red}{1.001} & \textcolor{black}{0.954} &  &  &  &  & \\
oct$(bdshr)$ & \textcolor{red}{1.002} & \textcolor{red}{1.110} & \textcolor{red}{1.071} & \textcolor{black}{0.989} & \textcolor{blue}{\textbf{0.941}} &  &  &  &  & \\
\bottomrule
\multicolumn{11}{l}{\rule{0pt}{1em}\rule{0pt}{2em}\textsuperscript{*}\makecell[l]{The Gaussian method employs a shrinkage covariance matrix and includes four techniques (G, B, H, HB)\\ with multi-step residuals.}}\\
\end{tabular}

\endgroup
\caption{CRPS skill score presented in equation (18) and (19) of the paper for the Australian Quarterly National Accounts dataset (AusGDP). The smaller this value, the more accurate the forecast. Approaches that performed worse than the benchmark model (Bootstrap base forecasts) are highlighted in red, the best for each column is marked in bold and in blue the lowest value.The notation used to refer to the reconciliation and base forecast samples is explained in more details in Section 8.1 of the paper.}
\end{table}

\begin{table}[H]
\centering
\begingroup
\spacingset{1}
\fontsize{9}{11}\selectfont

\begin{tabular}[t]{c|>{}cccc>{}c|ccccc}
\toprule
\multicolumn{1}{c}{\textbf{}} & \multicolumn{10}{c}{\textbf{Base forecasts' sample approach}} \\
\cmidrule(l{0pt}r{0pt}){2-11}
\multicolumn{1}{c}{\makecell[c]{\bfseries Reconciliation\\\bfseries approach}} & \multicolumn{1}{c}{ctjb} & \multicolumn{4}{c}{\makecell[c]{Gaussian approach\textsuperscript{*}}} & \multicolumn{1}{c}{ctjb} & \multicolumn{4}{c}{\makecell[c]{Gaussian approach\textsuperscript{*}}} \\
\multicolumn{1}{c}{} &  & G & B & H & \multicolumn{1}{c}{HB} &  & G & B & H & HB\\
\midrule
\addlinespace[0.3em]
\multicolumn{1}{c}{} & \multicolumn{5}{c}{\textbf{$\forall k \in \{12,6,4,3,2,1\}$}} & \multicolumn{5}{c}{\textbf{$k = 1$}}\\
base & \textcolor{black}{\textbf{1.000}} & \textcolor{black}{\textbf{0.958}} & \textcolor{black}{0.984} & \textcolor{black}{\textbf{0.972}} & \textcolor{black}{0.992} & \textcolor{black}{\textbf{1.000}} & \textcolor{black}{\textbf{0.954}} & \textcolor{black}{0.958} & \textcolor{black}{\textbf{0.954}} & \textcolor{black}{0.958}\\
ct$(bu)$ & \textcolor{red}{2.427} & \textcolor{red}{1.040} & \textcolor{red}{1.042} & \textcolor{red}{1.040} & \textcolor{red}{1.041} & \textcolor{red}{1.759} & \textcolor{red}{1.001} & \textcolor{red}{1.002} & \textcolor{red}{1.002} & \textcolor{red}{1.002}\\
ct$(shr_{cs}, bu_{te})$ & \textcolor{red}{1.243} & \textcolor{black}{0.988} & \textcolor{black}{\textbf{0.913}} & \textcolor{black}{0.990} & \textcolor{blue}{\textbf{0.913}} & \textcolor{red}{1.098} & \textcolor{red}{1.011} & \textcolor{black}{\textbf{0.938}} & \textcolor{red}{1.013} & \textcolor{blue}{\textbf{0.938}}\\
ct$(wls_{cs}, bu_{te})$ & \textcolor{red}{1.348} & \textcolor{red}{1.020} & \textcolor{black}{0.940} & \textcolor{red}{1.022} & \textcolor{black}{0.939} & \textcolor{red}{1.151} & \textcolor{red}{1.028} & \textcolor{black}{0.950} & \textcolor{red}{1.030} & \textcolor{black}{0.950}\\
oct$(wlsv)$ & \textcolor{red}{1.132} & \textcolor{red}{1.137} & \textcolor{red}{1.065} & \textcolor{red}{1.059} & \textcolor{black}{0.969} & \textcolor{red}{1.050} & \textcolor{red}{1.078} & \textcolor{black}{0.989} & \textcolor{red}{1.043} & \textcolor{black}{0.960}\\
oct$(bdshr)$ & \textcolor{red}{1.047} & \textcolor{red}{1.085} & \textcolor{red}{1.013} & \textcolor{red}{1.011} & \textcolor{black}{0.927} & \textcolor{red}{1.009} & \textcolor{red}{1.050} & \textcolor{black}{0.966} & \textcolor{red}{1.019} & \textcolor{black}{0.942}\\
\addlinespace[0.3em]
\multicolumn{1}{c}{} & \multicolumn{5}{c}{\textbf{$k = 2$}} & \multicolumn{5}{c}{\textbf{$k = 3$}}\\
base & \textcolor{black}{\textbf{1.000}} & \textcolor{black}{\textbf{0.960}} & \textcolor{black}{0.971} & \textcolor{black}{\textbf{0.958}} & \textcolor{black}{0.972} & \textcolor{black}{\textbf{1.000}} & \textcolor{black}{\textbf{0.963}} & \textcolor{black}{0.981} & \textcolor{black}{\textbf{0.966}} & \textcolor{black}{0.986}\\
ct$(bu)$ & \textcolor{red}{2.176} & \textcolor{red}{1.035} & \textcolor{red}{1.036} & \textcolor{red}{1.035} & \textcolor{red}{1.035} & \textcolor{red}{2.428} & \textcolor{red}{1.042} & \textcolor{red}{1.044} & \textcolor{red}{1.042} & \textcolor{red}{1.043}\\
ct$(shr_{cs}, bu_{te})$ & \textcolor{red}{1.192} & \textcolor{red}{1.020} & \textcolor{black}{\textbf{0.942}} & \textcolor{red}{1.021} & \textcolor{blue}{\textbf{0.942}} & \textcolor{red}{1.245} & \textcolor{red}{1.009} & \textcolor{black}{\textbf{0.931}} & \textcolor{red}{1.011} & \textcolor{blue}{\textbf{0.931}}\\
ct$(wls_{cs}, bu_{te})$ & \textcolor{red}{1.275} & \textcolor{red}{1.045} & \textcolor{black}{0.962} & \textcolor{red}{1.047} & \textcolor{black}{0.962} & \textcolor{red}{1.348} & \textcolor{red}{1.038} & \textcolor{black}{0.955} & \textcolor{red}{1.040} & \textcolor{black}{0.954}\\
oct$(wlsv)$ & \textcolor{red}{1.110} & \textcolor{red}{1.149} & \textcolor{red}{1.065} & \textcolor{red}{1.070} & \textcolor{black}{0.979} & \textcolor{red}{1.142} & \textcolor{red}{1.160} & \textcolor{red}{1.082} & \textcolor{red}{1.073} & \textcolor{black}{0.981}\\
oct$(bdshr)$ & \textcolor{red}{1.045} & \textcolor{red}{1.105} & \textcolor{red}{1.024} & \textcolor{red}{1.033} & \textcolor{black}{0.949} & \textcolor{red}{1.060} & \textcolor{red}{1.109} & \textcolor{red}{1.032} & \textcolor{red}{1.029} & \textcolor{black}{0.943}\\
\addlinespace[0.3em]
\multicolumn{1}{c}{} & \multicolumn{5}{c}{\textbf{$k = 4$}} & \multicolumn{5}{c}{\textbf{$k = 6$}}\\
base & \textcolor{black}{\textbf{1.000}} & \textcolor{black}{\textbf{0.962}} & \textcolor{black}{0.987} & \textcolor{black}{\textbf{0.973}} & \textcolor{black}{0.996} & \textcolor{black}{\textbf{1.000}} & \textcolor{black}{\textbf{0.963}} & \textcolor{black}{0.998} & \textcolor{black}{\textbf{0.984}} & \textcolor{red}{1.011}\\
ct$(bu)$ & \textcolor{red}{2.585} & \textcolor{red}{1.052} & \textcolor{red}{1.054} & \textcolor{red}{1.053} & \textcolor{red}{1.053} & \textcolor{red}{2.849} & \textcolor{red}{1.083} & \textcolor{red}{1.085} & \textcolor{red}{1.083} & \textcolor{red}{1.084}\\
ct$(shr_{cs}, bu_{te})$ & \textcolor{red}{1.277} & \textcolor{red}{1.000} & \textcolor{blue}{\textbf{0.923}} & \textcolor{red}{1.002} & \textcolor{black}{\textbf{0.923}} & \textcolor{red}{1.339} & \textcolor{black}{0.999} & \textcolor{black}{\textbf{0.921}} & \textcolor{red}{1.000} & \textcolor{blue}{\textbf{0.920}}\\
ct$(wls_{cs}, bu_{te})$ & \textcolor{red}{1.392} & \textcolor{red}{1.032} & \textcolor{black}{0.950} & \textcolor{red}{1.035} & \textcolor{black}{0.950} & \textcolor{red}{1.473} & \textcolor{red}{1.038} & \textcolor{black}{0.954} & \textcolor{red}{1.040} & \textcolor{black}{0.954}\\
oct$(wlsv)$ & \textcolor{red}{1.157} & \textcolor{red}{1.167} & \textcolor{red}{1.097} & \textcolor{red}{1.075} & \textcolor{black}{0.982} & \textcolor{red}{1.192} & \textcolor{red}{1.187} & \textcolor{red}{1.124} & \textcolor{red}{1.090} & \textcolor{black}{0.995}\\
oct$(bdshr)$ & \textcolor{red}{1.065} & \textcolor{red}{1.112} & \textcolor{red}{1.041} & \textcolor{red}{1.025} & \textcolor{black}{0.939} & \textcolor{red}{1.084} & \textcolor{red}{1.121} & \textcolor{red}{1.058} & \textcolor{red}{1.029} & \textcolor{black}{0.940}\\
\addlinespace[0.3em]
\multicolumn{1}{c}{} & \multicolumn{5}{c}{\textbf{$k = 12$}} & \multicolumn{5}{c}{}\\
base & \textcolor{black}{\textbf{1.000}} & \textcolor{black}{0.948} & \textcolor{red}{1.010} & \textcolor{red}{1.002} & \textcolor{red}{1.033} &  &  &  &  & \\
ct$(bu)$ & \textcolor{red}{2.990} & \textcolor{red}{1.028} & \textcolor{red}{1.031} & \textcolor{red}{1.029} & \textcolor{red}{1.029} &  &  &  &  & \\
ct$(shr_{cs}, bu_{te})$ & \textcolor{red}{1.326} & \textcolor{black}{\textbf{0.897}} & \textcolor{black}{\textbf{0.830}} & \textcolor{black}{\textbf{0.899}} & \textcolor{blue}{\textbf{0.830}} &  &  &  &  & \\
ct$(wls_{cs}, bu_{te})$ & \textcolor{red}{1.477} & \textcolor{black}{0.942} & \textcolor{black}{0.872} & \textcolor{black}{0.945} & \textcolor{black}{0.870} &  &  &  &  & \\
oct$(wlsv)$ & \textcolor{red}{1.149} & \textcolor{red}{1.089} & \textcolor{red}{1.041} & \textcolor{red}{1.006} & \textcolor{black}{0.922} &  &  &  &  & \\
oct$(bdshr)$ & \textcolor{red}{1.021} & \textcolor{red}{1.015} & \textcolor{black}{0.964} & \textcolor{black}{0.935} & \textcolor{black}{0.855} &  &  &  &  & \\
\bottomrule
\multicolumn{11}{l}{\rule{0pt}{1em}\rule{0pt}{2em}\textsuperscript{*}\makecell[l]{The Gaussian method employs a shrinkage covariance matrix and includes four techniques (G, B, H, HB)\\ with multi-step residuals.}}\\
\end{tabular}

\endgroup
\caption{ES skill score presented in equation (18) and (19) of the paper for the Australian Quarterly National Accounts dataset (AusGDP). The smaller this value, the more accurate the forecast. Approaches that performed worse than the benchmark model (Bootstrap base forecasts) are highlighted in red, the best for each column is marked in bold and in blue the lowest value. The notation used to refer to the reconciliation and base forecast samples is explained in more details in Section 8.1 of the paper.}
\end{table}


\begin{table}[H]
\centering
\begingroup
\spacingset{1}
\fontsize{9}{11}\selectfont

\begin{tabular}[t]{>{\centering\arraybackslash}p{2.5cm}>{\centering\arraybackslash}p{1.5cm}>{\centering\arraybackslash}p{1.5cm}>{\centering\arraybackslash}p{1.5cm}>{\centering\arraybackslash}p{1.5cm}>{\centering\arraybackslash}p{1.5cm}}
\toprule
\multicolumn{1}{c}{\textbf{}} & \multicolumn{5}{c}{\textbf{Base forecasts' sample approach}} \\
\cmidrule(l{0pt}r{0pt}){2-6}
\multicolumn{1}{c}{} & \multicolumn{1}{c}{} & \multicolumn{4}{c}{\makecell[c]{Gaussian approach: sample covariance matrix}} \\
\multicolumn{1}{c}{\makecell[c]{\bfseries Reconciliation\\\bfseries approach}} & \multicolumn{1}{c}{ctjb} & \multicolumn{2}{c}{Multi-step residuals} & \multicolumn{2}{c}{\makecell[c]{Overlapping and\\ multi-step residuals}} \\
\multicolumn{1}{c}{} &  & G & \multicolumn{1}{c}{H} & G & H\\
\midrule
\addlinespace[0.3em]
\multicolumn{6}{c}{\textbf{$\forall k \in \{4,2,1\}$}}\\
base & \textcolor{black}{1.000} & \textcolor{black}{0.979} & \textcolor{black}{0.995} & \textcolor{black}{0.968} & \textcolor{black}{0.976}\\
ct$(shr_{cs}, bu_{te})$ & \textcolor{black}{0.937} & \textcolor{black}{0.956} & \textcolor{black}{0.956} & \textcolor{black}{0.976} & \textcolor{black}{0.976}\\
ct$(wls_{cs}, bu_{te})$ & \textcolor{black}{0.930} & \textcolor{black}{0.917} & \textcolor{black}{0.917} & \textcolor{black}{0.898} & \textcolor{black}{0.898}\\
oct$(wlsv)$ & \textcolor{black}{0.926} & \textcolor{black}{0.919} & \textcolor{black}{0.920} & \textcolor{black}{0.900} & \textcolor{black}{0.900}\\
oct$(bdshr)$ & \textcolor{black}{0.940} & \textcolor{black}{0.965} & \textcolor{black}{0.945} & \textcolor{black}{0.992} & \textcolor{black}{0.957}\\
oct$(shr)$ & \textcolor{black}{0.944} & \textcolor{red}{1.020} & \textcolor{black}{0.940} & \textcolor{red}{1.094} & \textcolor{black}{0.988}\\
oct$(hshr)$ & \textcolor{black}{0.988} & \textcolor{black}{0.972} & \textcolor{red}{1.002} & \textcolor{black}{0.974} & \textcolor{red}{1.001}\\
oct$_o(wlsv)$ & \textcolor{black}{\textbf{0.926}} & \textcolor{black}{\textbf{0.911}} & \textcolor{black}{\textbf{0.912}} & \textcolor{black}{\textbf{0.896}} & \textcolor{blue}{\textbf{0.895}}\\
oct$_o(bdshr)$ & \textcolor{black}{0.978} & \textcolor{black}{0.964} & \textcolor{black}{0.946} & \textcolor{black}{0.952} & \textcolor{black}{0.930}\\
oct$_o(shr)$ & \textcolor{black}{0.950} & \textcolor{black}{0.946} & \textcolor{black}{0.922} & \textcolor{black}{0.925} & \textcolor{black}{0.903}\\
oct$_o(hshr)$ & \textcolor{black}{0.989} & \textcolor{black}{0.966} & \textcolor{black}{0.984} & \textcolor{black}{0.954} & \textcolor{black}{0.965}\\
oct$_{oh}(shr)$ & \textcolor{red}{1.102} & \textcolor{red}{1.059} & \textcolor{red}{1.001} & \textcolor{red}{1.094} & \textcolor{black}{0.988}\\
oct$_{oh}(hshr)$ & \textcolor{red}{1.006} & \textcolor{black}{0.983} & \textcolor{red}{1.009} & \textcolor{black}{0.974} & \textcolor{red}{1.001}\\
\addlinespace[0.3em]
\multicolumn{6}{c}{\textbf{$k = 1$}}\\
base & \textcolor{black}{1.000} & \textcolor{black}{0.988} & \textcolor{black}{0.988} & \textcolor{black}{0.971} & \textcolor{black}{0.971}\\
ct$(shr_{cs}, bu_{te})$ & \textcolor{black}{0.992} & \textcolor{red}{1.008} & \textcolor{red}{1.008} & \textcolor{red}{1.029} & \textcolor{red}{1.029}\\
ct$(wls_{cs}, bu_{te})$ & \textcolor{black}{0.986} & \textcolor{black}{0.974} & \textcolor{black}{0.975} & \textcolor{black}{0.956} & \textcolor{black}{0.956}\\
oct$(wlsv)$ & \textcolor{black}{0.984} & \textcolor{black}{0.981} & \textcolor{black}{0.979} & \textcolor{black}{0.959} & \textcolor{black}{0.959}\\
oct$(bdshr)$ & \textcolor{black}{0.997} & \textcolor{red}{1.019} & \textcolor{red}{1.003} & \textcolor{red}{1.044} & \textcolor{red}{1.018}\\
oct$(shr)$ & \textcolor{red}{1.015} & \textcolor{red}{1.095} & \textcolor{red}{1.010} & \textcolor{red}{1.160} & \textcolor{red}{1.059}\\
oct$(hshr)$ & \textcolor{red}{1.048} & \textcolor{red}{1.037} & \textcolor{red}{1.060} & \textcolor{red}{1.034} & \textcolor{red}{1.061}\\
oct$_o(wlsv)$ & \textcolor{black}{\textbf{0.984}} & \textcolor{black}{\textbf{0.971}} & \textcolor{black}{\textbf{0.970}} & \textcolor{black}{\textbf{0.954}} & \textcolor{blue}{\textbf{0.954}}\\
oct$_o(bdshr)$ & \textcolor{red}{1.034} & \textcolor{red}{1.016} & \textcolor{red}{1.003} & \textcolor{red}{1.005} & \textcolor{black}{0.989}\\
oct$_o(shr)$ & \textcolor{red}{1.014} & \textcolor{red}{1.003} & \textcolor{black}{0.985} & \textcolor{black}{0.987} & \textcolor{black}{0.968}\\
oct$_o(hshr)$ & \textcolor{red}{1.047} & \textcolor{red}{1.028} & \textcolor{red}{1.038} & \textcolor{red}{1.012} & \textcolor{red}{1.023}\\
oct$_{oh}(shr)$ & \textcolor{red}{1.172} & \textcolor{red}{1.109} & \textcolor{red}{1.066} & \textcolor{red}{1.160} & \textcolor{red}{1.059}\\
oct$_{oh}(hshr)$ & \textcolor{red}{1.068} & \textcolor{red}{1.046} & \textcolor{red}{1.059} & \textcolor{red}{1.034} & \textcolor{red}{1.061}\\
\addlinespace[0.3em]
\multicolumn{6}{c}{\textbf{$k = 2$}}\\
base & \textcolor{black}{1.000} & \textcolor{black}{0.984} & \textcolor{black}{0.993} & \textcolor{black}{0.968} & \textcolor{black}{0.976}\\
ct$(shr_{cs}, bu_{te})$ & \textcolor{black}{0.949} & \textcolor{black}{0.966} & \textcolor{black}{0.966} & \textcolor{black}{0.987} & \textcolor{black}{0.987}\\
ct$(wls_{cs}, bu_{te})$ & \textcolor{black}{0.942} & \textcolor{black}{0.928} & \textcolor{black}{0.928} & \textcolor{black}{0.909} & \textcolor{black}{0.909}\\
oct$(wlsv)$ & \textcolor{black}{0.938} & \textcolor{black}{0.929} & \textcolor{black}{0.931} & \textcolor{black}{0.911} & \textcolor{black}{0.911}\\
oct$(bdshr)$ & \textcolor{black}{0.953} & \textcolor{black}{0.976} & \textcolor{black}{0.956} & \textcolor{red}{1.003} & \textcolor{black}{0.969}\\
oct$(shr)$ & \textcolor{black}{0.955} & \textcolor{red}{1.031} & \textcolor{black}{0.951} & \textcolor{red}{1.113} & \textcolor{red}{1.002}\\
oct$(hshr)$ & \textcolor{red}{1.001} & \textcolor{black}{0.985} & \textcolor{red}{1.014} & \textcolor{black}{0.987} & \textcolor{red}{1.016}\\
oct$_o(wlsv)$ & \textcolor{black}{\textbf{0.938}} & \textcolor{black}{\textbf{0.921}} & \textcolor{black}{\textbf{0.923}} & \textcolor{black}{\textbf{0.907}} & \textcolor{blue}{\textbf{0.906}}\\
oct$_o(bdshr)$ & \textcolor{black}{0.991} & \textcolor{black}{0.974} & \textcolor{black}{0.957} & \textcolor{black}{0.964} & \textcolor{black}{0.942}\\
oct$_o(shr)$ & \textcolor{black}{0.965} & \textcolor{black}{0.958} & \textcolor{black}{0.934} & \textcolor{black}{0.938} & \textcolor{black}{0.916}\\
oct$_o(hshr)$ & \textcolor{red}{1.002} & \textcolor{black}{0.979} & \textcolor{black}{0.996} & \textcolor{black}{0.967} & \textcolor{black}{0.978}\\
oct$_{oh}(shr)$ & \textcolor{red}{1.120} & \textcolor{red}{1.069} & \textcolor{red}{1.013} & \textcolor{red}{1.113} & \textcolor{red}{1.002}\\
oct$_{oh}(hshr)$ & \textcolor{red}{1.021} & \textcolor{black}{0.996} & \textcolor{red}{1.021} & \textcolor{black}{0.987} & \textcolor{red}{1.016}\\
\addlinespace[0.3em]
\multicolumn{6}{c}{\textbf{$k = 4$}}\\
base & \textcolor{black}{1.000} & \textcolor{black}{0.966} & \textcolor{red}{1.004} & \textcolor{black}{0.964} & \textcolor{black}{0.981}\\
ct$(shr_{cs}, bu_{te})$ & \textcolor{black}{0.874} & \textcolor{black}{0.896} & \textcolor{black}{0.896} & \textcolor{black}{0.914} & \textcolor{black}{0.914}\\
ct$(wls_{cs}, bu_{te})$ & \textcolor{black}{0.866} & \textcolor{black}{0.853} & \textcolor{black}{0.853} & \textcolor{black}{0.834} & \textcolor{black}{0.834}\\
oct$(wlsv)$ & \textcolor{black}{0.860} & \textcolor{black}{0.853} & \textcolor{black}{0.855} & \textcolor{black}{0.835} & \textcolor{black}{0.834}\\
oct$(bdshr)$ & \textcolor{black}{0.874} & \textcolor{black}{0.904} & \textcolor{black}{0.880} & \textcolor{black}{0.931} & \textcolor{black}{0.889}\\
oct$(shr)$ & \textcolor{black}{0.866} & \textcolor{black}{0.940} & \textcolor{black}{0.864} & \textcolor{red}{1.015} & \textcolor{black}{0.909}\\
oct$(hshr)$ & \textcolor{black}{0.919} & \textcolor{black}{0.900} & \textcolor{black}{0.935} & \textcolor{black}{0.904} & \textcolor{black}{0.931}\\
oct$_o(wlsv)$ & \textcolor{black}{\textbf{0.860}} & \textcolor{black}{\textbf{0.847}} & \textcolor{black}{\textbf{0.848}} & \textcolor{black}{\textbf{0.832}} & \textcolor{blue}{\textbf{0.830}}\\
oct$_o(bdshr)$ & \textcolor{black}{0.914} & \textcolor{black}{0.905} & \textcolor{black}{0.883} & \textcolor{black}{0.892} & \textcolor{black}{0.865}\\
oct$_o(shr)$ & \textcolor{black}{0.877} & \textcolor{black}{0.882} & \textcolor{black}{0.852} & \textcolor{black}{0.854} & \textcolor{black}{0.831}\\
oct$_o(hshr)$ & \textcolor{black}{0.922} & \textcolor{black}{0.898} & \textcolor{black}{0.923} & \textcolor{black}{0.888} & \textcolor{black}{0.898}\\
oct$_{oh}(shr)$ & \textcolor{red}{1.020} & \textcolor{red}{1.002} & \textcolor{black}{0.928} & \textcolor{red}{1.015} & \textcolor{black}{0.909}\\
oct$_{oh}(hshr)$ & \textcolor{black}{0.934} & \textcolor{black}{0.912} & \textcolor{black}{0.951} & \textcolor{black}{0.904} & \textcolor{black}{0.931}\\
\bottomrule
\end{tabular}

\endgroup
\caption{CRPS}
\end{table}

\begin{table}[H]
\centering
\begingroup
\spacingset{1}
\fontsize{9}{11}\selectfont

\begin{tabular}[t]{l|>{}cccc>{}c|ccccc}
\toprule
\multicolumn{1}{c}{\textbf{}} & \multicolumn{10}{c}{\textbf{Generation of the base forecasts paths}} \\
\cmidrule(l{0pt}r{0pt}){2-11}
\multicolumn{1}{c}{\makecell[c]{\bfseries Reconciliation\\\bfseries approach}} & \multicolumn{1}{c}{ctjb} & \multicolumn{4}{c}{\makecell[c]{Gaussian approach\textsuperscript{*}}} & \multicolumn{1}{c}{ctjb} & \multicolumn{4}{c}{\makecell[c]{Gaussian approach\textsuperscript{*}}} \\
\multicolumn{1}{c}{} &  & G$_{h}$ & H$_{h}$ & G$_{oh}$ & \multicolumn{1}{c}{H$_{oh}$} &  & G$_{h}$ & H$_{h}$ & G$_{oh}$ & \multicolumn{1}{c}{H$_{oh}$}\\
\midrule
\addlinespace[0.3em]
\multicolumn{1}{c}{} & \multicolumn{5}{c}{\textbf{$\forall k \in \{4,2,1\}$}} & \multicolumn{5}{c}{\textbf{$k = 1$}}\\
base & \textcolor{black}{1.000} & \textcolor{black}{0.979} & \textcolor{red}{1.011} & \textcolor{black}{0.968} & \textcolor{black}{0.987} & \textcolor{black}{1.000} & \textcolor{black}{0.988} & \textcolor{black}{0.988} & \textcolor{black}{0.971} & \textcolor{black}{0.971}\\
ct$(shr_{cs}, bu_{te})$ & \textcolor{black}{0.937} & \textcolor{black}{0.960} & \textcolor{black}{0.961} & \textcolor{black}{0.962} & \textcolor{black}{0.960} & \textcolor{black}{0.992} & \textcolor{red}{1.001} & \textcolor{red}{1.001} & \textcolor{red}{1.004} & \textcolor{black}{1.000}\\
ct$(wls_{cs}, bu_{te})$ & \textcolor{black}{0.930} & \textcolor{black}{\textbf{0.951}} & \textcolor{black}{0.953} & \textcolor{blue}{\textbf{0.911}} & \textcolor{black}{0.915} & \textcolor{black}{0.986} & \textcolor{black}{0.997} & \textcolor{black}{0.998} & \textcolor{blue}{\textbf{0.964}} & \textcolor{black}{0.967}\\
oct$(wlsv)$ & \textcolor{black}{0.926} & \textcolor{black}{0.972} & \textcolor{black}{0.957} & \textcolor{black}{0.918} & \textcolor{black}{0.917} & \textcolor{black}{0.984} & \textcolor{red}{1.010} & \textcolor{red}{1.003} & \textcolor{black}{0.971} & \textcolor{black}{0.970}\\
oct$(bdshr)$ & \textcolor{black}{0.940} & \textcolor{black}{0.986} & \textcolor{black}{0.966} & \textcolor{black}{0.981} & \textcolor{black}{0.956} & \textcolor{black}{0.997} & \textcolor{red}{1.015} & \textcolor{red}{1.006} & \textcolor{red}{1.016} & \textcolor{black}{1.000}\\
oct$(shr)$ & \textcolor{black}{0.944} & \textcolor{black}{0.999} & \textcolor{black}{0.962} & \textcolor{red}{1.051} & \textcolor{black}{0.995} & \textcolor{red}{1.015} & \textcolor{red}{1.047} & \textcolor{red}{1.021} & \textcolor{red}{1.105} & \textcolor{red}{1.058}\\
oct$(hshr)$ & \textcolor{black}{0.988} & \textcolor{black}{1.000} & \textcolor{red}{1.021} & \textcolor{black}{0.979} & \textcolor{red}{1.002} & \textcolor{red}{1.048} & \textcolor{red}{1.045} & \textcolor{red}{1.066} & \textcolor{red}{1.034} & \textcolor{red}{1.053}\\
oct$_o(wlsv)$ & \textcolor{black}{\textbf{0.926}} & \textcolor{black}{0.961} & \textcolor{black}{0.948} & \textcolor{black}{0.914} & \textcolor{black}{\textbf{0.912}} & \textcolor{black}{\textbf{0.984}} & \textcolor{black}{1.000} & \textcolor{black}{0.993} & \textcolor{black}{0.966} & \textcolor{black}{\textbf{0.965}}\\
oct$_o(bdshr)$ & \textcolor{black}{0.978} & \textcolor{black}{0.956} & \textcolor{black}{0.949} & \textcolor{black}{0.949} & \textcolor{black}{0.934} & \textcolor{red}{1.034} & \textcolor{black}{\textbf{0.984}} & \textcolor{black}{\textbf{0.983}} & \textcolor{black}{0.988} & \textcolor{black}{0.977}\\
oct$_o(shr)$ & \textcolor{black}{0.950} & \textcolor{black}{0.957} & \textcolor{black}{\textbf{0.946}} & \textcolor{black}{0.933} & \textcolor{black}{0.917} & \textcolor{red}{1.014} & \textcolor{black}{0.998} & \textcolor{black}{0.995} & \textcolor{black}{0.986} & \textcolor{black}{0.974}\\
oct$_o(hshr)$ & \textcolor{black}{0.989} & \textcolor{black}{0.997} & \textcolor{red}{1.013} & \textcolor{black}{0.967} & \textcolor{black}{0.982} & \textcolor{red}{1.047} & \textcolor{red}{1.039} & \textcolor{red}{1.054} & \textcolor{red}{1.019} & \textcolor{red}{1.032}\\
oct$_{oh}(shr)$ & \textcolor{red}{1.102} & \textcolor{red}{1.010} & \textcolor{red}{1.006} & \textcolor{red}{1.051} & \textcolor{black}{0.995} & \textcolor{red}{1.172} & \textcolor{red}{1.059} & \textcolor{red}{1.063} & \textcolor{red}{1.105} & \textcolor{red}{1.058}\\
oct$_{oh}(hshr)$ & \textcolor{red}{1.006} & \textcolor{black}{0.989} & \textcolor{red}{1.004} & \textcolor{black}{0.979} & \textcolor{red}{1.002} & \textcolor{red}{1.068} & \textcolor{red}{1.037} & \textcolor{red}{1.050} & \textcolor{red}{1.034} & \textcolor{red}{1.053}\\
\addlinespace[0.3em]
\multicolumn{1}{c}{} & \multicolumn{5}{c}{\textbf{$k = 2$}} & \multicolumn{5}{c}{\textbf{$k = 4$}}\\
base & \textcolor{black}{1.000} & \textcolor{black}{0.984} & \textcolor{red}{1.009} & \textcolor{black}{0.968} & \textcolor{black}{0.987} & \textcolor{black}{1.000} & \textcolor{black}{0.966} & \textcolor{red}{1.037} & \textcolor{black}{0.964} & \textcolor{red}{1.002}\\
ct$(shr_{cs}, bu_{te})$ & \textcolor{black}{0.949} & \textcolor{black}{0.972} & \textcolor{black}{0.972} & \textcolor{black}{0.974} & \textcolor{black}{0.971} & \textcolor{black}{0.874} & \textcolor{black}{0.910} & \textcolor{black}{0.911} & \textcolor{black}{0.910} & \textcolor{black}{0.910}\\
ct$(wls_{cs}, bu_{te})$ & \textcolor{black}{0.942} & \textcolor{black}{\textbf{0.962}} & \textcolor{black}{0.964} & \textcolor{blue}{\textbf{0.923}} & \textcolor{black}{0.927} & \textcolor{black}{0.866} & \textcolor{black}{\textbf{0.897}} & \textcolor{black}{0.900} & \textcolor{black}{\textbf{0.851}} & \textcolor{black}{0.855}\\
oct$(wlsv)$ & \textcolor{black}{0.938} & \textcolor{black}{0.988} & \textcolor{black}{0.968} & \textcolor{black}{0.931} & \textcolor{black}{0.929} & \textcolor{black}{0.860} & \textcolor{black}{0.921} & \textcolor{black}{0.903} & \textcolor{black}{0.856} & \textcolor{black}{0.856}\\
oct$(bdshr)$ & \textcolor{black}{0.953} & \textcolor{red}{1.004} & \textcolor{black}{0.979} & \textcolor{black}{0.996} & \textcolor{black}{0.970} & \textcolor{black}{0.874} & \textcolor{black}{0.942} & \textcolor{black}{0.914} & \textcolor{black}{0.932} & \textcolor{black}{0.900}\\
oct$(shr)$ & \textcolor{black}{0.955} & \textcolor{red}{1.016} & \textcolor{black}{0.973} & \textcolor{red}{1.070} & \textcolor{red}{1.010} & \textcolor{black}{0.866} & \textcolor{black}{0.937} & \textcolor{black}{0.895} & \textcolor{black}{0.981} & \textcolor{black}{0.922}\\
oct$(hshr)$ & \textcolor{red}{1.001} & \textcolor{red}{1.015} & \textcolor{red}{1.034} & \textcolor{black}{0.993} & \textcolor{red}{1.017} & \textcolor{black}{0.919} & \textcolor{black}{0.942} & \textcolor{black}{0.965} & \textcolor{black}{0.913} & \textcolor{black}{0.937}\\
oct$_o(wlsv)$ & \textcolor{black}{\textbf{0.938}} & \textcolor{black}{0.976} & \textcolor{black}{\textbf{0.959}} & \textcolor{black}{0.927} & \textcolor{black}{\textbf{0.925}} & \textcolor{black}{\textbf{0.860}} & \textcolor{black}{0.910} & \textcolor{black}{0.894} & \textcolor{black}{0.853} & \textcolor{black}{0.852}\\
oct$_o(bdshr)$ & \textcolor{black}{0.991} & \textcolor{black}{0.970} & \textcolor{black}{0.963} & \textcolor{black}{0.963} & \textcolor{black}{0.948} & \textcolor{black}{0.914} & \textcolor{black}{0.917} & \textcolor{black}{0.905} & \textcolor{black}{0.899} & \textcolor{black}{0.880}\\
oct$_o(shr)$ & \textcolor{black}{0.965} & \textcolor{black}{0.973} & \textcolor{black}{0.959} & \textcolor{black}{0.948} & \textcolor{black}{0.931} & \textcolor{black}{0.877} & \textcolor{black}{0.903} & \textcolor{black}{\textbf{0.886}} & \textcolor{black}{0.868} & \textcolor{blue}{\textbf{0.850}}\\
oct$_o(hshr)$ & \textcolor{red}{1.002} & \textcolor{red}{1.013} & \textcolor{red}{1.026} & \textcolor{black}{0.980} & \textcolor{black}{0.996} & \textcolor{black}{0.922} & \textcolor{black}{0.943} & \textcolor{black}{0.962} & \textcolor{black}{0.905} & \textcolor{black}{0.921}\\
oct$_{oh}(shr)$ & \textcolor{red}{1.120} & \textcolor{red}{1.026} & \textcolor{red}{1.019} & \textcolor{red}{1.070} & \textcolor{red}{1.010} & \textcolor{red}{1.020} & \textcolor{black}{0.947} & \textcolor{black}{0.939} & \textcolor{black}{0.981} & \textcolor{black}{0.922}\\
oct$_{oh}(hshr)$ & \textcolor{red}{1.021} & \textcolor{red}{1.005} & \textcolor{red}{1.017} & \textcolor{black}{0.993} & \textcolor{red}{1.017} & \textcolor{black}{0.934} & \textcolor{black}{0.929} & \textcolor{black}{0.946} & \textcolor{black}{0.913} & \textcolor{black}{0.937}\\
\bottomrule
\multicolumn{11}{l}{\rule{0pt}{1em}\rule{0pt}{1.75em}\makecell[l]{$^\ast$The Gaussian method employs a shrinkage covariance matrix:\\G$_{h}$ and H$_{h}$ use multi-step residuals and G$_{oh}$ and H$_{oh}$ use overlapping and multi-step residuals.}}\\
\end{tabular}

\endgroup
\caption{CRPS}
\end{table}

\begin{table}[H]
\centering
\begingroup
\spacingset{1}
\fontsize{9}{11}\selectfont

\begin{tabular}[t]{c|>{}cccc>{}c|ccccc}
\toprule
\multicolumn{1}{c}{\textbf{}} & \multicolumn{10}{c}{\textbf{Base forecasts' sample approach}} \\
\cmidrule(l{0pt}r{0pt}){2-11}
\multicolumn{1}{c}{\makecell[c]{\bfseries Reconciliation\\\bfseries approach}} & \multicolumn{1}{c}{ctjb} & \multicolumn{4}{c}{\makecell[c]{Gaussian approach\textsuperscript{*}}} & \multicolumn{1}{c}{ctjb} & \multicolumn{4}{c}{\makecell[c]{Gaussian approach\textsuperscript{*}}} \\
\multicolumn{1}{c}{} &  & G$_{h}$ & H$_{h}$ & G$_{oh}$ & \multicolumn{1}{c}{H$_{oh}$} &  & G$_{h}$ & H$_{h}$ & G$_{oh}$ & \multicolumn{1}{c}{H$_{oh}$}\\
\midrule
\addlinespace[0.3em]
\multicolumn{1}{c}{} & \multicolumn{5}{c}{\textbf{$\forall k \in \{4,2,1\}$}} & \multicolumn{5}{c}{\textbf{$k = 1$}}\\
base & \textcolor{black}{1.000} & \textcolor{black}{0.970} & \textcolor{black}{0.988} & \textcolor{black}{0.960} & \textcolor{black}{0.970} & \textcolor{black}{1.000} & \textcolor{black}{0.977} & \textcolor{black}{0.977} & \textcolor{black}{0.965} & \textcolor{black}{0.965}\\
ct$(shr_{cs}, bu_{te})$ & \textcolor{black}{0.897} & \textcolor{black}{0.944} & \textcolor{black}{0.944} & \textcolor{black}{0.973} & \textcolor{black}{0.973} & \textcolor{black}{0.964} & \textcolor{red}{1.001} & \textcolor{red}{1.001} & \textcolor{red}{1.033} & \textcolor{red}{1.033}\\
ct$(wls_{cs}, bu_{te})$ & \textcolor{black}{\textbf{0.886}} & \textcolor{black}{0.880} & \textcolor{black}{0.880} & \textcolor{black}{\textbf{0.860}} & \textcolor{black}{0.860} & \textcolor{black}{\textbf{0.954}} & \textcolor{black}{\textbf{0.944}} & \textcolor{black}{0.945} & \textcolor{blue}{\textbf{0.928}} & \textcolor{black}{\textbf{0.928}}\\
oct$(wlsv)$ & \textcolor{black}{0.890} & \textcolor{black}{0.890} & \textcolor{black}{0.894} & \textcolor{black}{0.872} & \textcolor{black}{0.872} & \textcolor{black}{0.958} & \textcolor{black}{0.957} & \textcolor{black}{0.957} & \textcolor{black}{0.938} & \textcolor{black}{0.939}\\
oct$(bdshr)$ & \textcolor{black}{0.905} & \textcolor{black}{0.956} & \textcolor{black}{0.934} & \textcolor{black}{0.992} & \textcolor{black}{0.954} & \textcolor{black}{0.972} & \textcolor{red}{1.014} & \textcolor{black}{0.994} & \textcolor{red}{1.048} & \textcolor{red}{1.018}\\
oct$(shr)$ & \textcolor{black}{0.895} & \textcolor{black}{0.979} & \textcolor{black}{0.895} & \textcolor{red}{1.053} & \textcolor{black}{0.944} & \textcolor{black}{0.973} & \textcolor{red}{1.060} & \textcolor{black}{0.969} & \textcolor{red}{1.121} & \textcolor{red}{1.015}\\
oct$(hshr)$ & \textcolor{black}{0.951} & \textcolor{black}{0.940} & \textcolor{black}{0.973} & \textcolor{black}{0.959} & \textcolor{black}{0.992} & \textcolor{red}{1.017} & \textcolor{red}{1.010} & \textcolor{red}{1.034} & \textcolor{red}{1.023} & \textcolor{red}{1.055}\\
oct$_o(wlsv)$ & \textcolor{black}{0.891} & \textcolor{black}{\textbf{0.879}} & \textcolor{black}{0.881} & \textcolor{black}{0.864} & \textcolor{black}{0.864} & \textcolor{black}{0.958} & \textcolor{black}{0.945} & \textcolor{black}{0.945} & \textcolor{black}{0.931} & \textcolor{black}{0.931}\\
oct$_o(bdshr)$ & \textcolor{black}{0.940} & \textcolor{black}{0.928} & \textcolor{black}{0.910} & \textcolor{black}{0.918} & \textcolor{black}{0.895} & \textcolor{red}{1.004} & \textcolor{black}{0.986} & \textcolor{black}{0.971} & \textcolor{black}{0.980} & \textcolor{black}{0.961}\\
oct$_o(shr)$ & \textcolor{black}{0.900} & \textcolor{black}{0.899} & \textcolor{black}{\textbf{0.876}} & \textcolor{black}{0.878} & \textcolor{blue}{\textbf{0.858}} & \textcolor{black}{0.973} & \textcolor{black}{0.963} & \textcolor{black}{\textbf{0.944}} & \textcolor{black}{0.949} & \textcolor{black}{0.930}\\
oct$_o(hshr)$ & \textcolor{black}{0.956} & \textcolor{black}{0.936} & \textcolor{black}{0.955} & \textcolor{black}{0.922} & \textcolor{black}{0.936} & \textcolor{red}{1.021} & \textcolor{red}{1.004} & \textcolor{red}{1.012} & \textcolor{black}{0.987} & \textcolor{black}{1.000}\\
oct$_{oh}(shr)$ & \textcolor{red}{1.059} & \textcolor{red}{1.015} & \textcolor{black}{0.956} & \textcolor{red}{1.053} & \textcolor{black}{0.945} & \textcolor{red}{1.130} & \textcolor{red}{1.063} & \textcolor{red}{1.019} & \textcolor{red}{1.121} & \textcolor{red}{1.016}\\
oct$_{oh}(hshr)$ & \textcolor{black}{0.986} & \textcolor{black}{0.968} & \textcolor{black}{0.999} & \textcolor{black}{0.959} & \textcolor{black}{0.992} & \textcolor{red}{1.053} & \textcolor{red}{1.034} & \textcolor{red}{1.049} & \textcolor{red}{1.024} & \textcolor{red}{1.055}\\
\addlinespace[0.3em]
\multicolumn{1}{c}{} & \multicolumn{5}{c}{\textbf{$k = 2$}} & \multicolumn{5}{c}{\textbf{$k = 4$}}\\
base & \textcolor{black}{1.000} & \textcolor{black}{0.972} & \textcolor{black}{0.985} & \textcolor{black}{0.959} & \textcolor{black}{0.969} & \textcolor{black}{1.000} & \textcolor{black}{0.959} & \textcolor{red}{1.000} & \textcolor{black}{0.957} & \textcolor{black}{0.976}\\
ct$(shr_{cs}, bu_{te})$ & \textcolor{black}{0.915} & \textcolor{black}{0.961} & \textcolor{black}{0.960} & \textcolor{black}{0.991} & \textcolor{black}{0.991} & \textcolor{black}{0.818} & \textcolor{black}{0.874} & \textcolor{black}{0.874} & \textcolor{black}{0.899} & \textcolor{black}{0.900}\\
ct$(wls_{cs}, bu_{te})$ & \textcolor{black}{\textbf{0.904}} & \textcolor{black}{0.896} & \textcolor{black}{\textbf{0.896}} & \textcolor{blue}{\textbf{0.877}} & \textcolor{black}{\textbf{0.877}} & \textcolor{black}{\textbf{0.807}} & \textcolor{black}{0.805} & \textcolor{black}{0.805} & \textcolor{black}{\textbf{0.782}} & \textcolor{black}{0.783}\\
oct$(wlsv)$ & \textcolor{black}{0.909} & \textcolor{black}{0.907} & \textcolor{black}{0.912} & \textcolor{black}{0.889} & \textcolor{black}{0.889} & \textcolor{black}{0.811} & \textcolor{black}{0.813} & \textcolor{black}{0.819} & \textcolor{black}{0.794} & \textcolor{black}{0.794}\\
oct$(bdshr)$ & \textcolor{black}{0.925} & \textcolor{black}{0.976} & \textcolor{black}{0.953} & \textcolor{red}{1.013} & \textcolor{black}{0.974} & \textcolor{black}{0.825} & \textcolor{black}{0.883} & \textcolor{black}{0.860} & \textcolor{black}{0.920} & \textcolor{black}{0.876}\\
oct$(shr)$ & \textcolor{black}{0.913} & \textcolor{red}{1.000} & \textcolor{black}{0.914} & \textcolor{red}{1.076} & \textcolor{black}{0.963} & \textcolor{black}{0.807} & \textcolor{black}{0.885} & \textcolor{black}{0.808} & \textcolor{black}{0.967} & \textcolor{black}{0.861}\\
oct$(hshr)$ & \textcolor{black}{0.973} & \textcolor{black}{0.960} & \textcolor{black}{0.993} & \textcolor{black}{0.978} & \textcolor{red}{1.014} & \textcolor{black}{0.871} & \textcolor{black}{0.856} & \textcolor{black}{0.897} & \textcolor{black}{0.881} & \textcolor{black}{0.913}\\
oct$_o(wlsv)$ & \textcolor{black}{0.908} & \textcolor{black}{\textbf{0.895}} & \textcolor{black}{0.898} & \textcolor{black}{0.881} & \textcolor{black}{0.882} & \textcolor{black}{0.812} & \textcolor{black}{\textbf{0.802}} & \textcolor{black}{0.806} & \textcolor{black}{0.786} & \textcolor{black}{0.786}\\
oct$_o(bdshr)$ & \textcolor{black}{0.960} & \textcolor{black}{0.947} & \textcolor{black}{0.929} & \textcolor{black}{0.938} & \textcolor{black}{0.915} & \textcolor{black}{0.860} & \textcolor{black}{0.856} & \textcolor{black}{0.836} & \textcolor{black}{0.841} & \textcolor{black}{0.816}\\
oct$_o(shr)$ & \textcolor{black}{0.921} & \textcolor{black}{0.919} & \textcolor{black}{0.896} & \textcolor{black}{0.898} & \textcolor{black}{0.878} & \textcolor{black}{0.814} & \textcolor{black}{0.821} & \textcolor{black}{\textbf{0.796}} & \textcolor{black}{0.794} & \textcolor{blue}{\textbf{0.775}}\\
oct$_o(hshr)$ & \textcolor{black}{0.977} & \textcolor{black}{0.956} & \textcolor{black}{0.976} & \textcolor{black}{0.942} & \textcolor{black}{0.957} & \textcolor{black}{0.876} & \textcolor{black}{0.854} & \textcolor{black}{0.882} & \textcolor{black}{0.844} & \textcolor{black}{0.856}\\
oct$_{oh}(shr)$ & \textcolor{red}{1.082} & \textcolor{red}{1.029} & \textcolor{black}{0.973} & \textcolor{red}{1.076} & \textcolor{black}{0.963} & \textcolor{black}{0.971} & \textcolor{black}{0.954} & \textcolor{black}{0.882} & \textcolor{black}{0.967} & \textcolor{black}{0.861}\\
oct$_{oh}(hshr)$ & \textcolor{red}{1.007} & \textcolor{black}{0.988} & \textcolor{red}{1.017} & \textcolor{black}{0.979} & \textcolor{red}{1.014} & \textcolor{black}{0.904} & \textcolor{black}{0.888} & \textcolor{black}{0.934} & \textcolor{black}{0.881} & \textcolor{black}{0.913}\\
\bottomrule
\multicolumn{11}{l}{\rule{0pt}{1em}\rule{0pt}{1.75em}\makecell[l]{$^\ast$The Gaussian method employs a sample covariance matrix:\\G$_{h}$ and H$_{h}$ use multi-step residuals and G$_{oh}$ and H$_{oh}$ use overlapping and multi-step residuals.}}\\
\end{tabular}

\endgroup
\caption{ES}
\end{table}

\begin{table}[H]
\centering
\begingroup
\spacingset{1}
\fontsize{9}{11}\selectfont

\begin{tabular}[t]{>{\centering\arraybackslash}p{2.5cm}>{\centering\arraybackslash}p{1.5cm}>{\centering\arraybackslash}p{1.5cm}>{\centering\arraybackslash}p{1.5cm}>{\centering\arraybackslash}p{1.5cm}>{\centering\arraybackslash}p{1.5cm}}
\toprule
\multicolumn{1}{c}{\textbf{}} & \multicolumn{5}{c}{\textbf{Base forecasts' sample approach}} \\
\cmidrule(l{0pt}r{0pt}){2-6}
\multicolumn{1}{c}{} & \multicolumn{1}{c}{} & \multicolumn{4}{c}{\makecell[c]{Gaussian approach: shrinkage covariance matrix}} \\
\multicolumn{1}{c}{\makecell[c]{\bfseries Reconciliation\\\bfseries approach}} & \multicolumn{1}{c}{ctjb} & \multicolumn{2}{c}{Multi-step residuals} & \multicolumn{2}{c}{\makecell[c]{Overlapping and\\ multi-step residuals}} \\
\multicolumn{1}{c}{} &  & G & \multicolumn{1}{c}{H} & G & H\\
\midrule
\addlinespace[0.3em]
\multicolumn{6}{c}{\textbf{$\forall k \in \{4,2,1\}$}}\\
base & \textcolor{black}{1.000} & \textcolor{black}{0.967} & \textcolor{red}{1.002} & \textcolor{black}{0.957} & \textcolor{black}{0.980}\\
ct$(shr_{cs}, bu_{te})$ & \textcolor{black}{0.897} & \textcolor{black}{0.968} & \textcolor{black}{0.969} & \textcolor{black}{0.963} & \textcolor{black}{0.962}\\
ct$(wls_{cs}, bu_{te})$ & \textcolor{black}{\textbf{0.886}} & \textcolor{black}{0.939} & \textcolor{black}{0.944} & \textcolor{blue}{\textbf{0.882}} & \textcolor{black}{0.888}\\
oct$(wlsv)$ & \textcolor{black}{0.890} & \textcolor{black}{0.966} & \textcolor{black}{0.959} & \textcolor{black}{0.897} & \textcolor{black}{0.901}\\
oct$(bdshr)$ & \textcolor{black}{0.905} & \textcolor{black}{0.997} & \textcolor{black}{0.981} & \textcolor{black}{0.986} & \textcolor{black}{0.960}\\
oct$(shr)$ & \textcolor{black}{0.895} & \textcolor{black}{0.979} & \textcolor{black}{0.945} & \textcolor{red}{1.021} & \textcolor{black}{0.962}\\
oct$(hshr)$ & \textcolor{black}{0.951} & \textcolor{black}{0.997} & \textcolor{red}{1.023} & \textcolor{black}{0.973} & \textcolor{red}{1.005}\\
oct$_o(wlsv)$ & \textcolor{black}{0.891} & \textcolor{black}{0.950} & \textcolor{black}{0.945} & \textcolor{black}{0.889} & \textcolor{black}{0.892}\\
oct$_o(bdshr)$ & \textcolor{black}{0.940} & \textcolor{black}{0.935} & \textcolor{black}{0.933} & \textcolor{black}{0.922} & \textcolor{black}{0.909}\\
oct$_o(shr)$ & \textcolor{black}{0.900} & \textcolor{black}{\textbf{0.935}} & \textcolor{black}{\textbf{0.928}} & \textcolor{black}{0.895} & \textcolor{black}{\textbf{0.884}}\\
oct$_o(hshr)$ & \textcolor{black}{0.956} & \textcolor{black}{0.997} & \textcolor{red}{1.015} & \textcolor{black}{0.945} & \textcolor{black}{0.965}\\
oct$_{oh}(shr)$ & \textcolor{red}{1.059} & \textcolor{black}{0.981} & \textcolor{black}{0.983} & \textcolor{red}{1.021} & \textcolor{black}{0.962}\\
oct$_{oh}(hshr)$ & \textcolor{black}{0.986} & \textcolor{black}{0.996} & \textcolor{red}{1.014} & \textcolor{black}{0.973} & \textcolor{red}{1.005}\\
\addlinespace[0.3em]
\multicolumn{6}{c}{\textbf{$k = 1$}}\\
base & \textcolor{black}{1.000} & \textcolor{black}{0.973} & \textcolor{black}{0.973} & \textcolor{black}{0.961} & \textcolor{black}{0.962}\\
ct$(shr_{cs}, bu_{te})$ & \textcolor{black}{0.964} & \textcolor{red}{1.012} & \textcolor{red}{1.012} & \textcolor{red}{1.009} & \textcolor{red}{1.004}\\
ct$(wls_{cs}, bu_{te})$ & \textcolor{black}{\textbf{0.954}} & \textcolor{black}{0.994} & \textcolor{black}{0.998} & \textcolor{blue}{\textbf{0.947}} & \textcolor{black}{0.952}\\
oct$(wlsv)$ & \textcolor{black}{0.958} & \textcolor{red}{1.017} & \textcolor{red}{1.012} & \textcolor{black}{0.960} & \textcolor{black}{0.965}\\
oct$(bdshr)$ & \textcolor{black}{0.972} & \textcolor{red}{1.031} & \textcolor{red}{1.021} & \textcolor{red}{1.024} & \textcolor{red}{1.005}\\
oct$(shr)$ & \textcolor{black}{0.973} & \textcolor{red}{1.041} & \textcolor{red}{1.011} & \textcolor{red}{1.083} & \textcolor{red}{1.028}\\
oct$(hshr)$ & \textcolor{red}{1.017} & \textcolor{red}{1.051} & \textcolor{red}{1.073} & \textcolor{red}{1.034} & \textcolor{red}{1.063}\\
oct$_o(wlsv)$ & \textcolor{black}{0.958} & \textcolor{red}{1.002} & \textcolor{black}{0.997} & \textcolor{black}{0.953} & \textcolor{black}{0.956}\\
oct$_o(bdshr)$ & \textcolor{red}{1.004} & \textcolor{black}{\textbf{0.965}} & \textcolor{black}{\textbf{0.964}} & \textcolor{black}{0.969} & \textcolor{black}{0.959}\\
oct$_o(shr)$ & \textcolor{black}{0.973} & \textcolor{black}{0.984} & \textcolor{black}{0.982} & \textcolor{black}{0.960} & \textcolor{black}{\textbf{0.950}}\\
oct$_o(hshr)$ & \textcolor{red}{1.021} & \textcolor{red}{1.049} & \textcolor{red}{1.062} & \textcolor{red}{1.007} & \textcolor{red}{1.024}\\
oct$_{oh}(shr)$ & \textcolor{red}{1.130} & \textcolor{red}{1.034} & \textcolor{red}{1.041} & \textcolor{red}{1.083} & \textcolor{red}{1.029}\\
oct$_{oh}(hshr)$ & \textcolor{red}{1.053} & \textcolor{red}{1.050} & \textcolor{red}{1.064} & \textcolor{red}{1.034} & \textcolor{red}{1.063}\\
\addlinespace[0.3em]
\multicolumn{6}{c}{\textbf{$k = 2$}}\\
base & \textcolor{black}{1.000} & \textcolor{black}{0.970} & \textcolor{black}{0.999} & \textcolor{black}{0.955} & \textcolor{black}{0.980}\\
ct$(shr_{cs}, bu_{te})$ & \textcolor{black}{0.915} & \textcolor{black}{0.987} & \textcolor{black}{0.988} & \textcolor{black}{0.983} & \textcolor{black}{0.982}\\
ct$(wls_{cs}, bu_{te})$ & \textcolor{black}{\textbf{0.904}} & \textcolor{black}{\textbf{0.958}} & \textcolor{black}{0.962} & \textcolor{blue}{\textbf{0.900}} & \textcolor{black}{0.906}\\
oct$(wlsv)$ & \textcolor{black}{0.909} & \textcolor{black}{0.988} & \textcolor{black}{0.979} & \textcolor{black}{0.916} & \textcolor{black}{0.920}\\
oct$(bdshr)$ & \textcolor{black}{0.925} & \textcolor{red}{1.024} & \textcolor{red}{1.005} & \textcolor{red}{1.010} & \textcolor{black}{0.984}\\
oct$(shr)$ & \textcolor{black}{0.913} & \textcolor{red}{1.006} & \textcolor{black}{0.967} & \textcolor{red}{1.045} & \textcolor{black}{0.982}\\
oct$(hshr)$ & \textcolor{black}{0.973} & \textcolor{red}{1.020} & \textcolor{red}{1.046} & \textcolor{black}{0.994} & \textcolor{red}{1.028}\\
oct$_o(wlsv)$ & \textcolor{black}{0.908} & \textcolor{black}{0.972} & \textcolor{black}{0.964} & \textcolor{black}{0.908} & \textcolor{black}{0.911}\\
oct$_o(bdshr)$ & \textcolor{black}{0.960} & \textcolor{black}{0.959} & \textcolor{black}{0.957} & \textcolor{black}{0.945} & \textcolor{black}{0.932}\\
oct$_o(shr)$ & \textcolor{black}{0.921} & \textcolor{black}{0.958} & \textcolor{black}{\textbf{0.950}} & \textcolor{black}{0.917} & \textcolor{black}{\textbf{0.905}}\\
oct$_o(hshr)$ & \textcolor{black}{0.977} & \textcolor{red}{1.021} & \textcolor{red}{1.038} & \textcolor{black}{0.966} & \textcolor{black}{0.987}\\
oct$_{oh}(shr)$ & \textcolor{red}{1.082} & \textcolor{red}{1.002} & \textcolor{red}{1.003} & \textcolor{red}{1.045} & \textcolor{black}{0.982}\\
oct$_{oh}(hshr)$ & \textcolor{red}{1.007} & \textcolor{red}{1.017} & \textcolor{red}{1.036} & \textcolor{black}{0.994} & \textcolor{red}{1.028}\\
\addlinespace[0.3em]
\multicolumn{6}{c}{\textbf{$k = 4$}}\\
base & \textcolor{black}{1.000} & \textcolor{black}{0.958} & \textcolor{red}{1.033} & \textcolor{black}{0.953} & \textcolor{black}{1.000}\\
ct$(shr_{cs}, bu_{te})$ & \textcolor{black}{0.818} & \textcolor{black}{0.909} & \textcolor{black}{0.910} & \textcolor{black}{0.902} & \textcolor{black}{0.902}\\
ct$(wls_{cs}, bu_{te})$ & \textcolor{black}{\textbf{0.807}} & \textcolor{black}{0.871} & \textcolor{black}{0.876} & \textcolor{black}{\textbf{0.805}} & \textcolor{black}{0.812}\\
oct$(wlsv)$ & \textcolor{black}{0.811} & \textcolor{black}{0.896} & \textcolor{black}{0.891} & \textcolor{black}{0.820} & \textcolor{black}{0.825}\\
oct$(bdshr)$ & \textcolor{black}{0.825} & \textcolor{black}{0.938} & \textcolor{black}{0.919} & \textcolor{black}{0.926} & \textcolor{black}{0.895}\\
oct$(shr)$ & \textcolor{black}{0.807} & \textcolor{black}{0.898} & \textcolor{black}{0.864} & \textcolor{black}{0.940} & \textcolor{black}{0.881}\\
oct$(hshr)$ & \textcolor{black}{0.871} & \textcolor{black}{0.924} & \textcolor{black}{0.954} & \textcolor{black}{0.897} & \textcolor{black}{0.929}\\
oct$_o(wlsv)$ & \textcolor{black}{0.812} & \textcolor{black}{0.882} & \textcolor{black}{0.876} & \textcolor{black}{0.812} & \textcolor{black}{0.816}\\
oct$_o(bdshr)$ & \textcolor{black}{0.860} & \textcolor{black}{0.884} & \textcolor{black}{0.879} & \textcolor{black}{0.857} & \textcolor{black}{0.841}\\
oct$_o(shr)$ & \textcolor{black}{0.814} & \textcolor{black}{\textbf{0.867}} & \textcolor{black}{\textbf{0.857}} & \textcolor{black}{0.815} & \textcolor{blue}{\textbf{0.803}}\\
oct$_o(hshr)$ & \textcolor{black}{0.876} & \textcolor{black}{0.926} & \textcolor{black}{0.949} & \textcolor{black}{0.868} & \textcolor{black}{0.889}\\
oct$_{oh}(shr)$ & \textcolor{black}{0.971} & \textcolor{black}{0.910} & \textcolor{black}{0.911} & \textcolor{black}{0.941} & \textcolor{black}{0.882}\\
oct$_{oh}(hshr)$ & \textcolor{black}{0.904} & \textcolor{black}{0.924} & \textcolor{black}{0.947} & \textcolor{black}{0.896} & \textcolor{black}{0.929}\\
\bottomrule
\end{tabular}

\endgroup
\caption{ES}
\end{table}

\end{document}